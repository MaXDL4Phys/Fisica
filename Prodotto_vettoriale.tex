

\documentclass{beamer}
\usetheme{Madrid}
\usecolortheme{seahorse} % Tema colori
\definecolor{peach}{RGB}{255, 128, 64}

\setbeamercolor{title}{fg=teal}
\setbeamercolor{frametitle}{ fg=teal}
\setbeamercolor{structure}{fg=peach}

% Pacchetti aggiuntivi
\usepackage{xcolor}
\usepackage[utf8]{inputenc}
\usepackage[italian]{babel}
\usepackage{amsmath, amssymb}
\usepackage{graphicx}
\usepackage{tikz}
\usetikzlibrary{calc} % Libreria per il calcolo delle coordinate
\usetikzlibrary{scopes}
\title{Determinante delle Matrici e Applicazioni}
\author{Prof. M.Bosetti}
\date{\today}

\begin{document}

\frame{\titlepage}

% Slide 1: Introduzione
\begin{frame}{Introduzione: definizione}
    \textbf{Determinante:} Il determinante è una funzione che associa a ogni matrice quadrata un numero reale (o complesso), utile per: 
    \begin{itemize}
        \item Verificare l'invertibilità di una matrice.
        \item Risolvere sistemi lineari.
        \item Calcolare il volume di parallelepipedi in spazi vettoriali.
    \end{itemize}
\end{frame}

% Slide 2: Determinante di una matrice 2x2
\begin{frame}{Determinante di una matrice $2 \times 2$}
    Per una matrice quadrata $A$ di ordine $2 \times 2$: 
$$
    A = \begin{bmatrix}
    a & b \\
    c & d
    \end{bmatrix}
$$
    Il determinante si calcola come:
$$
    \det(A) = ad - bc
$$
    \textbf{Esempio:}
$$
    A = \begin{bmatrix}
    3 & 2 \\
    1 & 4
    \end{bmatrix}, \quad \det(A) = (3)(4) - (2)(1) = 12 - 2 = 10
$$
\end{frame}

% Slide 3: Determinante di una matrice 3x3
\begin{frame}{Determinante di una matrice $3 \times 3$}
    Per una matrice quadrata $A$ di ordine $3 \times 3$:
$$
    A = \begin{bmatrix}
    a & b & c \\
    d & e & f \\
    g & h & i
    \end{bmatrix}
$$
    Il determinante si calcola come:
$$
    \text{det}(A) = a(ei - fh) - b(di - fg) + c(dh - eg)
$$
    \textbf{Esempio:}
$$
    A = \begin{bmatrix}
    1 & 2 & 3 \\
    4 & 5 & 6 \\
    7 & 8 & 9
    \end{bmatrix}
$$
$$
    \text{det}(A) = 1(5 \cdot 9 - 6 \cdot 8) - 2(4 \cdot 9 - 6 \cdot 7) + 3(4 \cdot 8 - 5 \cdot 7)
$$
$$
    = 1(-3) - 2(-6) + 3(-4) = -3 + 12 - 12 = -3
$$
\end{frame}

% Slide 4: Prodotto vettoriale
\begin{frame}{Esempio: Prodotto vettoriale e determinante}
    \textbf{Definizione:} Il prodotto vettoriale di due vettori $\vec{u}$ e $\vec{v}$ in $\mathbb{R}^3$ può essere espresso tramite il determinante di una matrice:
$$
    \vec{u} \times \vec{v} = \begin{vmatrix}
    \hat{i} & \hat{j} & \hat{k} \\
    u_x & u_y & u_z \\
    v_x & v_y & v_z
    \end{vmatrix}
$$
    \textbf{Esempio:}
$$
    \vec{u} = (1, 2, 3), \quad \vec{v} = (4, 5, 6)
$$
    Calcoliamo:
$$
    \vec{u} \times \vec{v} = \hat{i} \begin{vmatrix}
    2 & 3 \\
    5 & 6
    \end{vmatrix}
    - \hat{j} \begin{vmatrix}
    1 & 3 \\
    4 & 6
    \end{vmatrix}
    + \hat{k} \begin{vmatrix}
    1 & 2 \\
    4 & 5
    \end{vmatrix}
$$
$$
    = \hat{i} (2 \cdot 6 - 3 \cdot 5) - \hat{j} (1 \cdot 6 - 3 \cdot 4) + \hat{k} (1 \cdot 5 - 2 \cdot 4)
$$
$$
    = -3\hat{i} + 6\hat{j} - 3\hat{k}
$$
\end{frame}

% Slide 5: Velocità e velocità angolare
\begin{frame}{Velocità lineare e velocità angolare: formula inversa}
    \textbf{Relazione generale:}
    La velocità lineare $\vec{v}$ è legata alla velocità angolare $\vec{\omega}$ tramite:
$$
    \vec{v} = \vec{\omega} \times \vec{r}
$$
    Per ricavare $\vec{\omega}$, usiamo:
$$
    \vec{\omega} = \frac{\vec{r} \times \vec{v}}{|\vec{r}|^2}
$$
    dove:
    \begin{itemize}
        \item $\vec{r} \times \vec{v}$: prodotto vettoriale tra $\vec{r}$ e $\vec{v}$,
        \item $|\vec{r}|^2 = r_x^2 + r_y^2 + r_z^2$.
    \end{itemize}
\end{frame}

% Slide 6: Esempio nel piano z-y
\begin{frame}{Esempio: Calcolo di $\vec{\omega}$ nel piano $z$-$y$}
    Supponiamo:
$$
    \vec{v} = (0, 12, 8), \quad \vec{r} = (0, 3, 4)
$$
    \textbf{Calcolo del prodotto vettoriale:}
$$
    \vec{r} \times \vec{v} = \begin{vmatrix}
    \hat{i} & \hat{j} & \hat{k} \\
    0 & 3 & 4 \\
    0 & 12 & 8
    \end{vmatrix}
$$
$$
    = \hat{i} (3 \cdot 8 - 4 \cdot 12) - \hat{j} (0 \cdot 8 - 4 \cdot 0) + \hat{k} (0 \cdot 12 - 3 \cdot 0)
$$
$$
    = -24\hat{i} + 0\hat{j} + 0\hat{k}
$$

    \textbf{Calcolo di $|\vec{r}|^2$:}
$$
    |\vec{r}|^2 = r_y^2 + r_z^2 = 3^2 + 4^2 = 9 + 16 = 25
$$

    \textbf{Calcolo di $\vec{\omega}$:}
$$
    \vec{\omega} = \frac{\vec{r} \times \vec{v}}{|\vec{r}|^2} = \frac{-24\hat{i}}{25} = -0.96\hat{i}
$$
    \textbf{Risultato:} La velocità angolare è:
$
    \vec{\omega} = -0.96\hat{i} \, \mathrm{rad/s} $

\end{frame}

% Slide: Caso piano x-y
\begin{frame}{Esempio: Calcolo di \(\vec{\omega}\) nel piano \(x\)-\(y\)}
    Supponiamo:
    \[
    \vec{v} = (v_x, v_y, 0), \quad \vec{r} = (r_x, r_y, 0)
    \]

    \textbf{Relazione:}
    La velocità angolare è:
    \[
    \vec{\omega} = \frac{\vec{r} \times \vec{v}}{|\vec{r}|^2}
    \]

    \textbf{Calcolo del prodotto vettoriale:}
    \[
    \vec{r} \times \vec{v} = \begin{vmatrix}
    \hat{i} & \hat{j} & \hat{k} \\
    r_x & r_y & 0 \\
    v_x & v_y & 0
    \end{vmatrix}
    \]
    Risolviamo:
    \[
    \vec{r} \times \vec{v} = \hat{i}(r_y \cdot 0 - 0 \cdot v_y) - \hat{j}(r_x \cdot 0 - 0 \cdot v_x) + \hat{k}(r_x \cdot v_y - r_y \cdot v_x)
    \]
    \[
    = \hat{i}(0) - \hat{j}(0) + \hat{k}(r_x v_y - r_y v_x)
    \]
    \[
    \vec{r} \times \vec{v} = (r_x v_y - r_y v_x) \hat{k}
    \]
    \end{frame}
    

    
    \begin{frame}
        
    \textbf{Calcolo di \(|\vec{r}|^2\):}
    \[
    |\vec{r}|^2 = r_x^2 + r_y^2
    \]

    \textbf{Velocità angolare:}
    La velocità angolare è quindi:
    \[
    \vec{\omega} = \frac{r_x v_y - r_y v_x}{r_x^2 + r_y^2} \hat{k}
    \]

    \textbf{Esempio numerico:}
    Supponiamo:
    \[
    \vec{v} = (3, 4, 0), \quad \vec{r} = (1, 2, 0)
    \]
    Calcoliamo:
    \[
    \vec{r} \times \vec{v} = \begin{vmatrix}
    \hat{i} & \hat{j} & \hat{k} \\
    1 & 2 & 0 \\
    3 & 4 & 0
    \end{vmatrix}
    = \hat{k}(1 \cdot 4 - 2 \cdot 3) = \hat{k}(4 - 6) = -2\hat{k}
    \]

    \(|\vec{r}|^2\):
    \[
    |\vec{r}|^2 = 1^2 + 2^2 = 1 + 4 = 5
    \]

    \textbf{Velocità angolare:}
    \[
    \vec{\omega} = \frac{-2}{5} \hat{k} = -0.4 \hat{k} \, \mathrm{rad/s}
    \]


\end{frame}

\end{document}
