\documentclass{beamer}

% Tema della presentazione
\usetheme{Madrid}

\usecolortheme{seahorse} % Tema colori
\definecolor{peach}{RGB}{255, 128, 64}

\setbeamercolor{title}{fg=teal}
\setbeamercolor{frametitle}{ fg=peach}
\setbeamercolor{structure}{fg=teal!40}

% Pacchetti utili
\usepackage[utf8]{inputenc}
\usepackage{amsmath, amssymb}
\usepackage{physics}
\usepackage{graphicx}
\usepackage{mathtools} % Per formule migliorate

% Titolo e Autore
\title{Introduzione ai Vettori e ai Versori}
% \subtitle{Uniforme e Uniformemente Accelerato}
\author{Prof. M.Bosetti}
\date{\today}

\begin{document}

% Slide del titolo
\begin{frame}
    \titlepage
\end{frame}

% Slide 1: Definizione di Vettore
\begin{frame}{Definizione di Vettore}
    \begin{itemize}
        \item Un \textbf{vettore} è un ente matematico definito da:
        \begin{itemize}
            \item \textbf{Modulo} (o lunghezza).
            \item \textbf{Direzione}.
            \item \textbf{Verso}.
        \end{itemize}
        \item Si rappresenta graficamente come una freccia:
        \begin{itemize}
            \item La lunghezza della freccia indica il modulo.
            \item La direzione è quella della freccia.
            \item La punta indica il verso.
        \end{itemize}
    \end{itemize}
    \vspace{0.5cm}
    \centering
    % \includegraphics[width=0.5\textwidth]{vector_example.png}
\end{frame}

% Slide 2: Rappresentazione Matematica
\begin{frame}{Rappresentazione Matematica}
    \begin{itemize}
        \item Un vettore può essere scritto in termini delle sue componenti lungo gli assi cartesiani in \textbf{forma colonna}:
        \[
        \vec{v} =
        \begin{pmatrix}
            v_x \\
            v_y
        \end{pmatrix}
        \quad \text{o in 3D: } \quad
        \vec{v} =
        \begin{pmatrix}
            v_x \\
            v_y \\
            v_z
        \end{pmatrix}
        \]
        \item Può anche essere scritto come combinazione lineare dei versori:
        \[
        \vec{v} = v_x
        \begin{pmatrix}
            1 \\
            0
        \end{pmatrix}
        + v_y
        \begin{pmatrix}
            0 \\
            1
        \end{pmatrix}
        \quad \text{(in 2D)}
        \]
        o
        \[
        \vec{v} = v_x
        \begin{pmatrix}
            1 \\
            0 \\
            0
        \end{pmatrix}
        + v_y
        \begin{pmatrix}
            0 \\
            1 \\
            0
        \end{pmatrix}
        + v_z
        \begin{pmatrix}
            0 \\
            0 \\
            1
        \end{pmatrix}
        \quad \text{(in 3D)}.
        \]
    \end{itemize}
\end{frame}

% Slide 3: Definizione di Versore
\begin{frame}{Definizione di Versore}
    \begin{itemize}
        \item Un \textbf{versore} è un vettore di modulo unitario ($|\hat{u}| = 1$).
        \item Serve per definire la direzione di un vettore.
        \item Dato un vettore $\vec{v}$, il versore associato è:
        \[
        \hat{v} = \frac{\vec{v}}{|\vec{v}|} \quad \text{dove} \quad
        |\vec{v}| = \sqrt{v_x^2 + v_y^2}.
        \]
        \item In 2D, i versori lungo gli assi cartesiani sono:
        \[
        \hat{i} =
        \begin{pmatrix}
            1 \\
            0
        \end{pmatrix}
        \quad \text{e} \quad
        \hat{j} =
        \begin{pmatrix}
            0 \\
            1
        \end{pmatrix}.
        \]
        \item In 3D, si aggiunge:
        \[
        \hat{i} =
        \begin{pmatrix}
            1 \\
            0 \\
            0
        \end{pmatrix}, \quad
        \hat{j} =
        \begin{pmatrix}
            0 \\
            1 \\
            0
        \end{pmatrix}, \quad
        \hat{k} =
        \begin{pmatrix}
            0 \\
            0 \\
            1
        \end{pmatrix}.
        \]
    \end{itemize}
\end{frame}

% Slide 4: Esempio Svolto
\begin{frame}{Esempio Svolto}
    \textbf{Esempio:} Trova il versore di $\vec{v} =
    \begin{pmatrix}
        3 \\
        4
    \end{pmatrix}$.
    \begin{itemize}
        \item Calcoliamo il modulo di $\vec{v}$:
        \[
        |\vec{v}| = \sqrt{3^2 + 4^2} = \sqrt{9 + 16} = \sqrt{25} = 5.
        \]
        \item Il versore associato è:
        \[
        \hat{v} = \frac{\vec{v}}{|\vec{v}|} =
        \frac{1}{5}
        \begin{pmatrix}
            3 \\
            4
        \end{pmatrix}
        =
        \begin{pmatrix}
            0.6 \\
            0.8
        \end{pmatrix}.
        \]
        \item Risultato: $\hat{v} =
        \begin{pmatrix}
            0.6 \\
            0.8
        \end{pmatrix}$.
    \end{itemize}
\end{frame}

% Slide Esempio Versori in 3D
\begin{frame}{Esempio: Versori in 3D}
    \textbf{Definizione:} I versori in 3D sono vettori di modulo unitario ($|\hat{u}| = 1$) lungo i tre assi cartesiani:
    % \vspace{0.5cm}
    \textbf{Esempio:} Scriviamo il vettore $\vec{v}$ come combinazione lineare dei versori:
    \[
    \vec{v} =
    \begin{pmatrix}
        3 \\
        -2 \\
        5
    \end{pmatrix}
    = 3 \hat{i} - 2 \hat{j} + 5 \hat{k} = 3  \begin{pmatrix}
        1 \\
        0 \\
        0
    \end{pmatrix} -2     \begin{pmatrix}
        0 \\
        1 \\
        0
    \end{pmatrix}  + 5     \begin{pmatrix}
        0 \\
        0 \\
        1
    \end{pmatrix}.
    \]

    % \vspace{0.5cm}
    \textbf{Calcolo del Versore Associato:}
    \begin{itemize}
        \item Il modulo di $\vec{v}$ è:
        \[
        |\vec{v}| = \sqrt{3^2 + (-2)^2 + 5^2} = \sqrt{9 + 4 + 25} = \sqrt{38}.
        \]
        \item Il versore associato è:
        \[
        \hat{v} = \frac{\vec{v}}{|\vec{v}|} =
        \frac{1}{\sqrt{38}}
        \begin{pmatrix}
            3 \\
            -2 \\
            5
        \end{pmatrix}
        =
        \begin{pmatrix}
            \frac{3}{\sqrt{38}} \\
            \frac{-2}{\sqrt{38}} \\
            \frac{5}{\sqrt{38}}
        \end{pmatrix}.
        \]
    \end{itemize}

    % \vspace{0.5cm}
    % \centering
    % \includegraphics[width=0.6\textwidth]{3d_vectors.png} % Inserire immagine di vettori in 3D
\end{frame}


% Slide 5: Esercizi Proposti
\begin{frame}{Esercizi Proposti}
    \begin{enumerate}
        \item Calcola il modulo dei seguenti vettori:
        \[
        \vec{v}_1 =
        \begin{pmatrix}
            1 \\
            2
        \end{pmatrix}, \quad
        \vec{v}_2 =
        \begin{pmatrix}
            0 \\
            -3
        \end{pmatrix}, \quad
        \vec{v}_3 =
        \begin{pmatrix}
            -2 \\
            2
        \end{pmatrix}.
        \]
        \item Trova i versori associati ai vettori sopra.
        \item Scrivi $\vec{v} =
        \begin{pmatrix}
            4 \\
            -2 \\
            3
        \end{pmatrix}$ come combinazione lineare di $\hat{i}$, $\hat{j}$, $\hat{k}$.
    \end{enumerate}
\end{frame}

% % Slide Finale
% \begin{frame}
%     \centering
%     \Huge Domande?
% \end{frame}

\end{document}
