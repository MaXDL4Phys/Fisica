

\documentclass{beamer}
\usetheme{Madrid}
\usecolortheme{seahorse} % Tema colori
\definecolor{peach}{RGB}{255, 128, 64}

\setbeamercolor{title}{fg=teal}
\setbeamercolor{frametitle}{ fg=teal}
\setbeamercolor{structure}{fg=peach}

% Pacchetti aggiuntivi
\usepackage{xcolor}
\usepackage[utf8]{inputenc}
\usepackage[italian]{babel}
\usepackage{amsmath, amssymb}
\usepackage{graphicx}
\usepackage{tikz}
\usetikzlibrary{calc} % Libreria per il calcolo delle coordinate
\usetikzlibrary{scopes}
\title{Moto Circolare}
\subtitle{Uniforme e Uniformemente Accelerato}
\author{Prof. M.Bosetti}
\date{\today}

\begin{document}

% Slide 1: Titolo
\begin{frame}
    \titlepage
\end{frame}

% Slide 2: Introduzione
\begin{frame}{Introduzione al Moto Circolare}
    \begin{itemize}
        \item Il moto circolare è un moto lungo una traiettoria circolare.
        \item Tipologie:
        \begin{enumerate}
            \item \textbf{Moto Circolare Uniforme (MCU)}: velocità angolare costante.
            \item \textbf{Moto Circolare Uniformemente Accelerato (MCUA)}: accelerazione angolare costante.
        \end{enumerate}
    \end{itemize}
    \vspace{1cm}
    % \includegraphics[width=0.8\textwidth]{placeholder.png} % Segnaposto per immagine
\end{frame}

% Slide 3: Parametri fondamentali
\begin{frame}{Parametri Fondamentali del Moto Circolare}
    \begin{itemize}
        \item \textbf{Raggio} (\(r\)): distanza dal centro alla traiettoria.
        \item \textbf{Velocità Angolare} (\(\omega\)): quanto velocemente varia l'angolo.
        \[ \omega = \frac{\Delta\theta}{\Delta t} = \frac{2\pi}{T}= 2\pi f  \quad (\text{rad/s}) \]
        \item \textbf{Accelerazione Angolare} (\(\alpha\)): variazione di \(\omega\) nel tempo.
        \[ \alpha = \frac{\Delta\omega}{\Delta t} \quad (\text{rad/s}^2) \]
        \item \textbf{Velocità Tangenziale} (\(v\)): velocità lineare lungo la circonferenza.
        \[ v = r \cdot \omega \]
        \item \textbf{Accelerazione Centripeta} (\(a_c\)): mantiene il corpo sulla traiettoria.
        \[ a_c = \frac{v^2}{r} = r \cdot \omega^2 \]
    \end{itemize}
\end{frame}

% Slide 4: Moto Circolare Uniforme
\begin{frame}{Moto Circolare Uniforme (MCU)}
    \begin{itemize}
        \item \textbf{Caratteristiche}:
        \begin{itemize}
            \item Velocità angolare \(\omega\): costante.
            \item Accelerazione angolare \(\alpha\): nulla.
            \item Moto periodico: tempo costante per completare un giro.
        \end{itemize}
        \item \textbf{Formule principali}:
        \[ \theta(t) = \omega t + \theta_0, \quad v = r \cdot \omega, \quad a_c = r \cdot \omega^2 \]
    \end{itemize}
    % \includegraphics[width=0.8\textwidth]{placeholder.png} % Segnaposto per immagine
\end{frame}

% Slide 5: Moto Circolare Uniformemente Accelerato
\begin{frame}{Moto Circolare Uniformemente Accelerato (MCUA)}
    \begin{itemize}
        \item \textbf{Caratteristiche}:
        \begin{itemize}
            \item Velocità angolare \(\omega\): varia nel tempo.
            \item Accelerazione angolare \(\alpha\): costante.
        \end{itemize}
        \item \textbf{Formule principali}:
        \[ \theta(t) = \theta_0 + \omega_0 t + \frac{1}{2}\alpha t^2 \]
        \[ \omega(t) = \omega_0 + \alpha t, \quad v(t) = r \cdot \omega(t),\]
        \[a_c = r \cdot \omega^2, \quad a_t = r \cdot \alpha\]
        $$ a = \sqrt{a_t^2 + a_c^2} = \sqrt{r^2 \cdot \omega^4 + r^2 \cdot \alpha^2 } = r \sqrt{  \omega^4 +\alpha^2 }  $$
    \end{itemize}
    % \includegraphics[width=0.8\textwidth]{placeholder.png} % Segnaposto per immagine
\end{frame}

% Slide 6: Tabella comparativa
% Slide 6: Tabella comparativa con dimensioni ridotte
\begin{frame}{Tabella Comparativa: MCU vs MCUA}
    \begin{table}[h!]
        \centering
        \resizebox{\textwidth}{!}{ % Ridimensiona la tabella per adattarla alla larghezza del frame
        \begin{tabular}{|l|c|c|}
            \hline
            \textbf{Parametro} & \textbf{MCU (Moto Circolare Uniforme)} & \textbf{MCUA (Moto Circolare Uniformemente Accelerato)} \\
            \hline
            Velocità Angolare (\(\omega\) ) & \(\omega = \text{costante}\) & \(\omega(t) = \omega_0 + \alpha t\) \\
            \hline
            Accelerazione Angolare (\(\alpha\)) & \(\alpha = 0\) & \(\alpha = \text{costante}\) \\
            \hline
            Posizione Angolare (\(\theta\)) & \(\theta(t) = \omega t + \theta_0\) & \(\theta(t) = \theta_0 + \omega_0 t + \frac{1}{2}\alpha t^2\) \\
            \hline
            Velocità Tangenziale (\(v\)) & \(v = r \cdot \omega\) & \(v(t) = r \cdot \omega(t)\) \\
            \hline
            Accelerazione Centripeta (\(a_c\)) & \(a_c = r \cdot \omega^2\) & \(a_c = r \cdot \omega^2\) \\
            \hline
            Accelerazione Tangenziale (\(a_t\)) & \(a_t = 0\) & \(a_t = r \cdot \alpha\) \\
            \hline
            Accelerazione Totale (\(a\)) & \(a = a_c = r \cdot \omega^2\) & \(a = \sqrt{a_c^2 + a_t^2}\) \\
            \hline
        \end{tabular}
        }
    \end{table}
\end{frame}

% Slide 7: Conclusione
\begin{frame}{Conclusione}
    \begin{itemize}
        \item Il moto circolare è fondamentale per descrivere fenomeni ciclici.
        \item Differenze principali tra MCU e MCUA risiedono nella variazione delle velocità e delle accelerazioni.
        \item Esempi pratici:
        \begin{itemize}
            \item MCU: rotazione di una ruota.
            \item MCUA: accelerazione di una giostra.
        \end{itemize}
    \end{itemize}
    % \includegraphics[width=0.8\textwidth]{placeholder.png} % Segnaposto per immagine
\end{frame}

\end{document}