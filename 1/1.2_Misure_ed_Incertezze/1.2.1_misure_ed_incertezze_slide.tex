\documentclass[9pt]{beamer}
\usetheme{Madrid}
\usecolortheme{seahorse} % Tema colori
\usepackage[dvipsnames]{xcolor}
% Pacchetti aggiuntivi
\usepackage{xcolor}

% Colori personalizzati
\setbeamercolor{title}{fg=teal}
\setbeamercolor{frametitle}{bg=teal!40, fg=violet}
\setbeamercolor{structure}{fg=teal}

% Definizione dei colori personalizzati per teoria e problemi
\definecolor{theorycolor}{RGB}{173, 216, 230} % Blu chiaro per la teoria
\definecolor{problemcolor}{RGB}{255, 100, 0}   % Arancione per i problemi
\title{Medie e semidisperzione}
\subtitle{}
\author{Prof. Massimo Bosetti}
\institute{Liceo da Vinci}
\date{}


\begin{document}
\frame{\titlepage}
% Slide 1: Media
\begin{frame}{Media di una serie di misure}
\textbf{Definizione:} La media aritmetica di una serie di misure si calcola sommando tutte le misure e dividendo per il numero totale delle misure:
$$
\bar{x} = \frac{\sum_{i=1}^N x_i}{N}
$$
\textbf{Esempio:}
\begin{itemize}
    \item Misure: $ 4.0, 4.2, 3.9, 4.1, 4.0 $
    \item Calcolo:
    $$
    \bar{x} = \frac{4.0 + 4.2 + 3.9 + 4.1 + 4.0}{5} = 4.04
    $$
\end{itemize}
\end{frame}

% Slide 2: Semidispersione
\begin{frame}{Semidispersione}
\textbf{Definizione:} La semidispersione rappresenta la metà della differenza tra il valore massimo e il valore minimo:
$$
\delta = \frac{x_{\text{MAX}} - x_{\text{min}}}{2}
$$
\textbf{Esempio:}
\begin{itemize}
    \item Misure: $ 4.0, 4.2, 3.9, 4.1, 4.0 $
    \item Valore massimo: $ x_{\text{MAX}} = 4.2 $, valore minimo: $ x_{\text{min}} = 3.9 $
    \item Calcolo:
    $$
    \delta = \frac{4.2 - 3.9}{2} = 0.15
    $$
\end{itemize}
\end{frame}

% Slide 3: Deviazione standard
\begin{frame}{Deviazione Standard}
\textbf{Definizione:} La deviazione standard misura quanto le misure si discostano dalla media:
$$
\sigma = \sqrt{\frac{\sum_{i=1}^N (x_i - \bar{x})^2}{N}}
$$
\textbf{Esempio:}
\begin{itemize}
    \item Misure: $ 4.0, 4.2, 3.9, 4.1, 4.0 $
    \item Media: $ \bar{x} = 4.04 $
    \item Calcolo:
    $$
    \sigma = \sqrt{\frac{(4.0 - 4.04)^2 + (4.2 - 4.04)^2 + \dots}{5}} = 0.1
    $$
\end{itemize}
\end{frame}

% Slide: Incertezza Relativa e Percentuale
\begin{frame}{Incertezza Relativa e Percentuale}
\textbf{Definizione:}
\begin{itemize}
    \item \textbf{Incertezza relativa:} Rapporto tra l’incertezza assoluta e la misura stessa:
    $$
    \delta_{\text{rel}} = \frac{\delta x}{x}
    $$
    \item \textbf{Incertezza percentuale:} L’incertezza relativa espressa in percentuale:
    $$
    \delta_{\text{perc}} = \delta_{\text{rel}} \cdot 100
    $$
\end{itemize}

\textbf{Esempio:}
\begin{itemize}
    \item Misura: $ x = 8.0 \pm 0.2 $
    \item Calcolo dell’incertezza relativa:
    $$
    \delta_{\text{rel}} = \frac{0.2}{8.0} = 0.025
    $$
    \item Calcolo dell’incertezza percentuale:
    $$
    \delta_{\%} = 0.025 \cdot 100 = 2.5\%
    $$
\end{itemize}

\textbf{Risultato:} La misura può essere espressa come:
$
x = 8.0 \pm 2.5\%
$
\end{frame}

\section{Propagazione degli errori}
\begin{frame}{Propagazione degli errori}
\centering
\Huge Propagazione degli errori
\end{frame}


% Slide 4: Propagazione degli errori
\begin{frame}{Propagazione degli errori}
\textbf{Definizione:} La propagazione degli errori permette di calcolare l’incertezza nei risultati ottenuti combinando misure incerte.
\begin{itemize}
    \item \textbf{Somma o differenza:}
    $$
    \Delta z = \sqrt{(\Delta x)^2 + (\Delta y)^2}
    $$
    \item \textbf{Moltiplicazione o divisione:}
    $$
    \frac{\Delta z}{z} = \sqrt{\left(\frac{\Delta x}{x}\right)^2 + \left(\frac{\Delta y}{y}\right)^2}
    $$
\end{itemize}
\textbf{Esempio:}
\begin{itemize}
    \item Misure: $ x = 4.0 \pm 0.2 $, $ y = 3.0 \pm 0.1 $
    \item Calcolo per $ z = x + y $:
    $$
    \Delta z = \sqrt{(0.2)^2 + (0.1)^2} = 0.22
    $$
    Risultato: $ z = 7.0 \pm 0.22 $
\end{itemize}
\end{frame}

% Slide 5: Esercizi Svolti
\begin{frame}{Esercizi Svolti}
\textbf{Esercizio 1:} Calcola la media, la semidispersione e la deviazione standard per:
$$
5.0, 4.8, 5.2, 5.1, 4.9
$$
\textbf{Esercizio 2:} Trova il risultato e l’errore per:
$$
z = x \cdot y, \quad x = 2.0 \pm 0.1, \quad y = 3.0 \pm 0.2
$$
\end{frame}

% Slide 6: Esercizi da Fare
\begin{frame}{Esercizi da Fare}
\textbf{Esercizio 1:}
\begin{itemize}
    \item Misure: $ 3.0, 3.1, 3.2, 3.0, 3.1 $
    \item Calcola: media, semidispersione, deviazione standard.
\end{itemize}
\textbf{Esercizio 2:}
\begin{itemize}
    \item Due grandezze: $ a = 4.0 \pm 0.1 $, $ b = 2.0 \pm 0.05 $
    \item Calcola $ c = a - b $ con il relativo errore.
\end{itemize}
\textbf{Esercizio 3:}
\begin{itemize}
    \item Misure combinate: $ x = 6.0 \pm 0.3 $, $ y = 2.0 \pm 0.2 $
    \item Calcola $ z = x / y $ con l’errore.
\end{itemize}
\end{frame}

\end{document}