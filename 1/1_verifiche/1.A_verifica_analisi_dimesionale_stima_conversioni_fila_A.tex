\documentclass[12pt]{article}
\usepackage[utf8]{inputenc}
\usepackage[margin=1in]{geometry}
\usepackage{amsmath}
\usepackage{enumitem} % Per personalizzare gli elenchi

\pagestyle{empty}

\begin{document}

% Fila B
\begin{center}
    \textbf{\Large Verifica di Fisica- Fila B}
    \vspace{0.25cm}
    
    Liceo Scientifico L. da Vinci - 2024/25
    \vspace{0.25cm}
    
\end{center}
Nome e Cognome: \_\_\_\_\_\_\_\_\_\_\_\_\_\_\_\_\_\_\_\_\_\_\_\_\_\_\_\_\_\_\_\_\_\_\_\_\_\_\_ 
 \hspace{4cm} Classe: 1\_\_
\vspace{0.5cm}
\hrule
\vspace{0.5cm}

\textbf{Istruzioni:} 
\begin{itemize}
    \item Utilizzare la notazione scientifica ove necessario.
    \item Spiegare chiaramente i passaggi.
    \item Tempo a disposizione: 50 minuti.
    \item nel foglio metterte il giusto numero del problema (1.a, 1.b etc)
\end{itemize}
\hrule
\vspace{0.5cm}

\section{Problemi alla Fermi}
\begin{enumerate}[label=\alph*)]
        \item Quanti cellulari sono venduti ogni anno nel mondo?
        \item Quanti litri d’acqua vengono usati per fare la doccia in un giorno in una grande città?
    \end{enumerate}


\section{Analisi Dimensionale}

    \begin{enumerate}[label=\alph*)]
        \item Determinare le dimensioni fisiche della pressione sapendo che $P = \frac{F}{A}$.
        \item Verificare dimensionalmente la formula della velocità media $v_m = \frac{s}{t}$.
        \item Determinare le dimensioni della potenza sapendo che $P = \frac{W}{t}$.
    \end{enumerate}

\section{Conversioni e Notazione Scientifica}

    \begin{enumerate}[label=\alph*)]
        \item Convertire $24400 \, \text{mm}$ in dam.
        \item Convertire $0.021 \, \text{ms}$ in $\mu$s.
        \item Scrivere in notazione scientifica: $0.00043$, $156$, $15000000000$.
        \item Convertire $650 \, \text{cm/s}$ in $\text{m/min}$.
        \item Una sostanza ha una densità di $2.7 \, \text{g/cm}^3$. Convertire questa densità in $\text{kg/m}^3$.
    \end{enumerate}

\end{document}