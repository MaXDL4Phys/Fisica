\documentclass[12pt]{article}
\usepackage[utf8]{inputenc}
\usepackage[margin=1in]{geometry}
\usepackage{amsmath}
\usepackage{enumitem} % Per personalizzare gli elenchi

\pagestyle{empty}

\begin{document}

% Fila B
\begin{center}
    \textbf{\Large Verifica di Fisica- Fila B}
    \vspace{0.25cm}
    
    Liceo Scientifico L. da Vinci - 2024/25
    \vspace{0.25cm}
    
\end{center}
Nome e Cognome: \_\_\_\_\_\_\_\_\_\_\_\_\_\_\_\_\_\_\_\_\_\_\_\_\_\_\_\_\_\_\_\_\_\_\_\_\_\_\_ 
 \hspace{4cm} Classe: 1\_\_
\vspace{0.5cm}
\hrule
\vspace{0.5cm}

\textbf{Istruzioni:} 
\begin{itemize}
    \item Utilizzare la notazione scientifica ove necessario.
    \item Spiegare chiaramente i passaggi.
    \item Tempo a disposizione: 50 minuti.
    \item nel foglio metterte il giusto numero del problema (1.a, 1.b etc)
\end{itemize}
\hrule
\vspace{0.5cm}

\section{Problemi alla Fermi}
    
    \begin{enumerate}[label=\alph*)]
        \item Quanti pezzi di pizza sono mangiati in Italia ogni giorno?
        \item Quante formiche ci sono in un parco di 1 km²?
    \end{enumerate}

\section{Analisi Dimensionale}
    \begin{enumerate}[label=\alph*)]
        \item Determinare le dimensioni fisiche dell'energia cinetica $E_k = \frac{1}{2}mv^2$.
        \item Verificare dimensionalmente la formula della forza gravitazionale $F = G\frac{m_1 m_2}{r^2}$.
        \item Determinare le dimensioni del lavoro sapendo che $W = F \cdot s$.
    \end{enumerate}
\section{Conversioni e Notazione Scientifica}
    \begin{enumerate}[label=\alph*)]
        \item Convertire $34 \, \text{mmol}$ in mol.
        \item Convertire $0.1 \, \text{hL}$ in cL.
        \item Scrivere in notazione scientifica: $0.076$, $1879$, $0.000000000012$.
        \item Convertire $120 \, \text{m/h}$ in $\text{km/s}$.
        \item Una sostanza ha una densità di $1.2 \, \text{g/cm}^3$. Convertire questa densità in $\text{kg/m}^3$.
    \end{enumerate}
\end{document}