\documentclass[9pt]{beamer}
\usetheme{Madrid}
\usecolortheme{seahorse} % Tema colori

% Pacchetti aggiuntivi
\usepackage{xcolor}

% Colori personalizzati
\setbeamercolor{title}{fg=teal}
\setbeamercolor{frametitle}{bg=teal!40, fg=white}
\setbeamercolor{structure}{fg=teal}


% Definizione dei colori personalizzati per teoria e problemi
\definecolor{theorycolor}{RGB}{173, 216, 230} % Blu chiaro per la teoria
\definecolor{problemcolor}{RGB}{255, 100, 0}   % Arancione per i problemi
\title{Diamo un po' di numeri: Analisi delle grandezze fisiche}
\subtitle{}
\author{Prof. Massimo Bosetti}
\institute{Liceo da Vinci}
\date{}

\begin{document}

% Titolo
\begin{frame}
    \titlepage
\end{frame}

% Indice
\begin{frame}{Indice}
    \tableofcontents
\end{frame}

% Sezione 1: Introduzione
\section{Introduzione}
\begin{frame}{Numeri e costanti fondamentali}
    \begin{itemize}
        \item Grandezze fondamentali per comprendere l'universo.
        \item Costanti non deducibili, ottenute tramite misurazioni.
        \item Importanza della gravità: ad esempio $g = 9.80 \, \mathrm{m/s^2}$.
    \end{itemize}
\end{frame}

% % Sezione 2: Unità di Misura Fondamentali
% \section{Unità di Misura Fondamentali}
% \begin{frame}{Unità di Misura Fondamentali nel SI}
%     \begin{table}[h!]
%         \centering
%         \begin{tabular}{|l|c|l|c|}
%             \hline
%             \textbf{Grandezza} & \textbf{Simbolo} & \textbf{Unità di Misura} & \textbf{Ordine di Grandezza} \\
%             \hline
%             Lunghezza & $L$ & metro ($\mathrm{m}$) & $10^{0}$ \\
%             Massa & $M$ & chilogrammo ($\mathrm{kg}$) & $10^{0}$ \\
%             Tempo & $T$ & secondo ($\mathrm{s}$) & $10^{0}$ \\
%             Corrente elettrica & $I$ & ampere ($\mathrm{A}$) & $10^{0}$ \\
%             Temperatura & $\Theta$ & kelvin ($\mathrm{K}$) & $10^{0}$ \\
%             Quantità di sostanza & $N$ & mole ($\mathrm{mol}$) & $10^{0}$ \\
%             Intensità luminosa & $J$ & candela ($\mathrm{cd}$) & $10^{0}$ \\
%             \hline
%         \end{tabular}
%         \caption{Unità di misura fondamentali del SI.}
%     \end{table}
% \end{frame}

% Sezione 3: Costanti Fisiche Fondamentali
\section{Costanti Fisiche Fondamentali}
\begin{frame}{Costanti Fisiche Fondamentali}
    \small
    \begin{table}[h!]
        \centering
        \begin{tabular}{|l|c|c|c|}
            \hline
            \textbf{Costante} & \textbf{Valore} & \textbf{Unità} & \textbf{Ordine di Grandezza} \\
            \hline
            Accelerazione di gravità ($g$) & $9.80665$ & $\mathrm{m/s^2}$ & $10^{1}$ \\
            Carica elementare ($e$) & $1.602 \times 10^{-19}$ & $\mathrm{C}$ & $10^{-19}$ \\
            Costante dei gas ($R$) & $8.31451$ & $\mathrm{J/mol \cdot K}$ & $10^{0}$ \\
            Costante di Boltzmann ($k$) & $1.38066 \times 10^{-23}$ & $\mathrm{J/K}$ & $10^{-23}$ \\
            Costante di Planck ($h$) & $6.626075 \times 10^{-34}$ & $\mathrm{J \cdot s}$ & $10^{-33}$ \\
            Costante dielettrica del vuoto ($\epsilon_0$) & $8.85419 \times 10^{-12}$ & $\mathrm{F/m}$ & $10^{-12}$ \\
            Costante gravitazionale ($G$) & $6.67259 \times 10^{-11}$ & $\mathrm{N \cdot m^2 / kg^2}$ & $10^{-10}$ \\
            Velocità della luce ($c$) & $2.99792458 \times 10^{8}$ & $\mathrm{m/s}$ & $10^{8}$ \\
            \hline
        \end{tabular}
        \caption{Costanti fisiche fondamentali.}
    \end{table}
\end{frame}

% Sezione 4: Fattori di Conversione
\section{Fattori di Conversione}
\begin{frame}{Fattori di Conversione}
    \begin{table}[h!]
        \centering
        \begin{tabular}{|l|c|c|c|}
            \hline
            \textbf{Unità} & \textbf{Simbolo} & \textbf{Fattore di Conversione} & \textbf{Ordine di Grandezza} \\
            \hline
            Angström & $\mathrm{\AA}$ & $10^{-10} \, \mathrm{m}$ & $10^{-10}$ \\
            Caloria & $\mathrm{cal}$ & $4.180 \, \mathrm{J}$ & $10^{0}$ \\
            Cavallo vapore & $\mathrm{CV}$ & $735.5 \, \mathrm{W}$ & $10^{3}$ \\
            Elettronvolt & $\mathrm{eV}$ & $1.602 \times 10^{-19} \, \mathrm{J}$ & $10^{-19}$ \\
            Parsec & $\mathrm{pc}$ & $3.086 \times 10^{16} \, \mathrm{m}$ & $10^{16}$ \\
            \hline
        \end{tabular}
        \caption{Fattori di conversione.}
    \end{table}
\end{frame}

% Sezione 5: Costanti Sperimentali di Uso Comune
\section{Costanti Sperimentali di Uso Comune}
\begin{frame}{Costanti Sperimentali di Uso Comune}
    \begin{table}[h!]
        \centering
        \begin{tabular}{|l|c|c|c|}
            \hline
            \textbf{Grandezza} & \textbf{Valore} & \textbf{Unità} & \textbf{Ordine di Grandezza} \\
            \hline
            Densità dell'acqua ($\rho_{\text{acqua}}$) & $1.00 \times 10^{3}$ & $\mathrm{kg/m^3}$ & $10^{3}$ \\
            Densità dell'alluminio ($\rho_{\text{Al}}$) & $2.70 \times 10^{3}$ & $\mathrm{kg/m^3}$ & $10^{3}$ \\
            Densità dell'aria ($\rho_{\text{aria}}$) & $1.29$ & $\mathrm{kg/m^3}$ & $10^{0}$ \\
            Massa della Terra ($M_{\text{T}}$) & $5.98 \times 10^{24}$ & $\mathrm{kg}$ & $10^{25}$ \\
            Massa del Sole ($M_{\text{S}}$) & $1.99 \times 10^{30}$ & $\mathrm{kg}$ & $10^{30}$ \\
            \hline
        \end{tabular}
        \caption{Costanti sperimentali di uso comune.}
    \end{table}
\end{frame}

% Sezione 6: Confronto tra Lunghezze, Tempi e Masse (completa)
\section{Confronto tra Lunghezze, Tempi e Masse}
\begin{frame}{Confronto tra Lunghezze, Tempi e Masse}
\small
    \begin{table}[h!]
        \centering
        \begin{tabular}{|l|c|c|c|}
            \hline
            \textbf{Tipo} & \textbf{Esempio} & \textbf{Valore} & \textbf{Ordine di Grandezza} \\
            \hline
            Lunghezza & Nucleo atomico & $10^{-15} \, \mathrm{m}$ & $10^{-15}$ \\
            Lunghezza & Atomo & $10^{-11} \, \mathrm{m}$ & $10^{-11}$ \\
            Lunghezza & DNA & $10^{-9} \, \mathrm{m}$ & $10^{-9}$ \\
            Lunghezza & Diametro terrestre & $10^{7} \, \mathrm{m}$ & $10^{7}$ \\
            Lunghezza & Distanza Terra-Sole & $10^{11} \, \mathrm{m}$ & $10^{11}$ \\
            Lunghezza & Raggio della galassia & $3 \times 10^{20} \, \mathrm{m}$ & $10^{20}$ \\
            \hline
            Tempo & Tempo minimo quantistico & $10^{-43} \, \mathrm{s}$ & $10^{-43}$ \\
            Tempo & Vita media del neutrone & $10^{3} \, \mathrm{s}$ & $10^{3}$ \\
            Tempo & Vita media del protone & $10^{31} \, \mathrm{s}$ & $10^{31}$ \\
            Tempo & Durata di un anno & $3 \times 10^{7} \, \mathrm{s}$ & $10^{7}$ \\
            Tempo & Età dell'universo & $5 \times 10^{17} \, \mathrm{s}$ & $10^{18}$ \\
            \hline
            Massa & Massa dell'elettrone & $10^{-30} \, \mathrm{kg}$ & $10^{-30}$ \\
            Massa & Massa dell'atomo di ferro & $10^{-25} \, \mathrm{kg}$ & $10^{-25}$ \\
            Massa & Massa di un uomo & $10^{2} \, \mathrm{kg}$ & $10^{2}$ \\
            Massa & Massa della Terra & $5.98 \times 10^{24} \, \mathrm{kg}$ & $10^{25}$ \\
            Massa & Massa dell'universo & $10^{51} \, \mathrm{kg}$ & $10^{51}$ \\
            \hline
        \end{tabular}
        \caption{Confronto tra lunghezze, tempi e masse basato sulla pagina 17 del documento.}
    \end{table}
\end{frame}

% Sezione 7: Problemi Proposti
\section{Problemi Proposti}

% Problema 1: Densità della Terra e confronto con Giove
\begin{frame}{Problema 1: Densità della Terra e confronto con Giove}
    \textbf{Obiettivo:} Determinare la densità della Terra e confrontarla con quella di Giove.
    
    \textbf{Formula:} $\rho = \frac{M}{V}, \quad V = \frac{4}{3} \pi R^3$
    
    \textbf{Calcolo per la Terra:}
    \[
    \rho_T = \frac{3 M_T}{4 \pi R_T^3} = \frac{3 \times 5.98 \times 10^{24}}{4 \pi (6.38 \times 10^6)^3} \approx 5.5 \times 10^3 \, \mathrm{kg/m^3}
    \]
    
    \textbf{Calcolo per Giove:}
    \[
    \rho_G = \frac{3 M_G}{4 \pi R_G^3} = \frac{3 \times 1.90 \times 10^{27}}{4 \pi (7.13 \times 10^7)^3} \approx 1.3 \times 10^3 \, \mathrm{kg/m^3}
    \]
    
    \textbf{Risultato:} La densità della Terra è circa 4 volte quella di Giove.
\end{frame}

% Problema 2: Volume occupato da una molecola d'acqua allo stato liquido
\begin{frame}{Problema 2: Volume occupato da una molecola d'acqua allo stato liquido}
    \textbf{Obiettivo:} Calcolare il volume occupato da una molecola d'acqua in fase liquida.
    
    \textbf{Dati:}
    \begin{itemize}
        \item Massa molare dell'acqua: $18 \, \mathrm{g/mol} = 1.8 \times 10^{-2} \, \mathrm{kg/mol}$
        \item Densità dell'acqua: $\rho = 10^3 \, \mathrm{kg/m^3}$
        \item Numero di Avogadro: $N_A = 6.022 \times 10^{23} \, \mathrm{mol^{-1}}$
    \end{itemize}
    
    \textbf{Calcolo:}
    \[
    V_{\text{molecola}} = \frac{M}{\rho N_A} = \frac{1.8 \times 10^{-2}}{10^3 \cdot 6.022 \times 10^{23}} \approx 3.0 \times 10^{-29} \, \mathrm{m^3}
    \]
    
    \textbf{Risultato:} Il volume di una molecola d'acqua è circa $3.0 \times 10^{-29} \, \mathrm{m^3}$.
\end{frame}

% Problema 3: Spazio occupato da una molecola d'acqua allo stato di vapore
\begin{frame}{Problema 3: Spazio occupato da una molecola d'acqua allo stato di vapore}
    \textbf{Obiettivo:} Determinare lo spazio disponibile per una molecola d'acqua in fase vapore a $100^\circ \mathrm{C}$.
    
    \textbf{Dati:}
    \begin{itemize}
        \item Densità del vapore acqueo: $\rho_{\text{vap}} = 0.60 \, \mathrm{kg/m^3}$
        \item Numero di Avogadro: $N_A = 6.022 \times 10^{23} \, \mathrm{mol^{-1}}$
    \end{itemize}
    
    \textbf{Calcolo:}
    \[
    V_{\text{vapore}} = \frac{M}{\rho_{\text{vap}} N_A} = \frac{1.8 \times 10^{-2}}{0.60 \cdot 6.022 \times 10^{23}} \approx 5.0 \times 10^{-25} \, \mathrm{m^3}
    \]
    
    \textbf{Rapporto con il liquido:}
    \[
    \frac{V_{\text{vapore}}}{V_{\text{liquido}}} \approx \frac{5.0 \times 10^{-25}}{3.0 \times 10^{-29}} \approx 1700
    \]
    
    \textbf{Risultato:} Una molecola occupa uno spazio circa 1700 volte maggiore in fase vapore rispetto alla fase liquida.
\end{frame}

% Problema 4: Densità del protone (continuazione)
\begin{frame}{Problema 4: Densità del protone (continuazione)}
    \textbf{Risultato:} La densità del protone è estremamente elevata, circa:
    \[
    \rho_p \approx 1.67 \times 10^{18} \, \mathrm{kg/m^3}
    \]
    Questo valore è nettamente superiore alla densità della materia conosciuta sulla Terra (ad esempio, l'oro ha una densità di $19.3 \times 10^{3} \, \mathrm{kg/m^3}$).

    \textbf{Conclusione:} La densità del protone riflette la natura estremamente compatta della materia nucleare.
\end{frame}

% Conclusioni
\section{Conclusioni}
\begin{frame}{Conclusioni}
    \textbf{Sintesi della presentazione:}
    \begin{itemize}
        \item Le costanti fisiche fondamentali, i fattori di conversione e le grandezze fisiche ci forniscono un quadro quantitativo essenziale dell'universo.
        \item I calcoli dimostrano come si possano derivare informazioni chiave utilizzando semplici formule e valori tabulati.
        \item La comprensione delle scale di grandezza (lunghezze, tempi e masse) è fondamentale per collocare i fenomeni fisici nel giusto contesto.
    \end{itemize}

    \textbf{Conclusione pratica:}
    \begin{itemize}
        \item Utilizzare le tabelle delle costanti consente di risolvere problemi scientifici con maggiore precisione e velocità.
        \item La comprensione delle proprietà delle particelle elementari e delle grandezze macroscopiche offre una visione completa della fisica.
    \end{itemize}
\end{frame}

\end{document}