\documentclass{beamer}
\usetheme{Madrid}
\usecolortheme{seahorse} % Tema colori

% Pacchetti aggiuntivi
\usepackage{xcolor}



% Colori personalizzati
\setbeamercolor{title}{fg=cyan}
\setbeamercolor{frametitle}{bg=blue!40, fg=white}
\setbeamercolor{structure}{fg=cyan}

\title{Analisi Dimensionale}
\subtitle{Appunti ed Esercizi} % Aggiunto sottotitolo
\author{Prof. Massimo Bosetti}
\institute{Liceo da Vinci }
\date{2024/25}

\begin{document}

\begin{frame}
    \titlepage
\end{frame}

\begin{frame}{Definizioni di Base}
    \begin{itemize}
        \item Le grandezze fisiche si dividono in \textbf{fondamentali} e \textbf{derivate}.
        \item Le grandezze fondamentali includono lunghezza, tempo e massa, misurabili direttamente.
        \item Le grandezze derivate si ottengono dalle fondamentali, es. superficie, volume.
    \end{itemize}
\end{frame}

\begin{frame}{Grandezze Fondamentali}
    \begin{table}[]
        \centering
        \begin{tabular}{|c|c|c|c|}
            \hline
            Nome & Simbolo & Unità (SI) & Dimensioni \\
            \hline
            Lunghezza & \( \ell \) & m & \([\ell]\) \\
            Tempo & \( t \) & s & \([t]\) \\
            Massa & \( m \) & kg & \([M]\) \\
            Corrente elettrica & \( I \) & A & \([I]\) \\
            Temperatura termodinamica & \( T \) & K & \([T]\) \\
            Quantità di sostanza & \( n \) & mol & \([N]\) \\
            Intensità luminosa & \( I_l \) & cd & \([J]\) \\
            \hline
        \end{tabular}
    \end{table}
\end{frame}


\begin{frame}{Esempio: Superficie}
    \begin{itemize}
        \item La superficie di un rettangolo si ottiene come prodotto di due lunghezze.
        \item Analisi dimensionale:
        \[
        [S] = [\ell] \cdot [\ell] = [\ell^2]
        \]
        \item Quindi, \( S \) è una grandezza derivata con dimensioni di una lunghezza al quadrato.
    \end{itemize}
\end{frame}

\begin{frame}{Problema 1: Volume di un Parallelepipedo}
    \begin{itemize}
        \item Formula del volume: \( V = a \cdot b \cdot c \)
        \item Analisi dimensionale:
        \[
        [V] = [a] \cdot [b] \cdot [c] = [\ell^3]
        \]
        \item Il volume ha quindi dimensioni di lunghezza al cubo.
    \end{itemize}
\end{frame}

\begin{frame}{Problema 2: Densità}
    \begin{itemize}
        \item Formula della densità: \( d = \frac{m}{V} \)
        \item Utilizzando \( [V] = [\ell^3] \) e \( [m] = m \):
        \[
        [d] = \frac{[m]}{[\ell^3]} = [m \cdot \ell^{-3}]
        \]
    \end{itemize}
\end{frame}

\begin{frame}{Problema 3: Uguaglianze Dimensionali}
    \begin{itemize}
        \item Verifica delle uguaglianze dimensionali:
        \[
        [d \cdot L] = [m \cdot \ell^{-2}]
        \]
        \item Confronto con l'uguaglianza:
        \[
        \text{Uguaglianza verificata}
        \]
    \end{itemize}
\end{frame}

\end{document}