\documentclass[a4paper,12pt]{article}
\usepackage[utf8]{inputenc}
\usepackage[italian]{babel}
\usepackage{amsmath}
\usepackage{amssymb}
\usepackage{geometry}
\geometry{a4paper, margin=2.5cm}

\title{Foglio di Esercizi - Equivalenze e Problemi}
\author{Prof. M.Bosetti}
\date{2025}

\begin{document}

\maketitle

\section*{Equivalenze e Notazione Scientifica}

\subsection*{Esercizio 1 (Equivalenze Semplici)}
Si eseguano le seguenti conversioni di unità di misura, specificando quale grandezza fisica viene espressa da ciascuna misura:
\begin{itemize}
    \item (a) Convertire \(24400 \, \text{mm}\) in \(\text{dam}\).
    \item (b) Convertire \(0.021 \, \text{ms}\) in \(\mu\text{s}\).
    \item (c) Convertire \(0.51 \, \text{Mg}\) in \(\text{kg}\).
    \item (d) Convertire \(34 \, \text{mmol}\) in \(\text{mol}\).
    \item (e) Convertire \(183000 \, \text{mcd}\) in \(\text{kcd}\).
    \item (f) Convertire \(12 \, \text{A}\) in \(\text{kA}\).
    \item (g) Convertire \(130 \, \text{ns}\) in \(\text{ms}\).
    \item (h) Convertire \(0.1 \, \text{hL}\) in \(\text{cL}\).
    \item (i) Convertire \(3762 \, \mu\text{m}\) in \(\text{cm}\).
    \item (j) Convertire \(0.014 \, \text{cs}\) in \(\text{s}\).
\end{itemize}

\subsection*{Esercizio 2 (Equivalenze Complicate)}
Si eseguano le seguenti conversioni di unità di misura, specificando quale grandezza fisica viene espressa da ciascuna misura:
\begin{itemize}
    \item (a) Convertire \(132 \, \text{cm}^2\) in \(\text{m}^2\).
    \item (b) Convertire \(0.00032 \, \text{m}^3\) in \(\text{mm}^3\).
    \item (c) Convertire \(400000 \, \text{m}^2\) in \(\text{km}^2\).
    \item (d) Convertire \(1500 \, \text{cm}^3\) in \(\text{m}^3\).
    \item (e) Convertire \(100 \, \text{km/h}\) in \(\text{m/s}\).
    \item (f) Convertire \(50 \, \text{m/s}\) in \(\text{km/h}\).
    \item (g) Convertire \(500 \, \text{L/h}\) in \(\text{L/min}\).
    \item (h) Convertire \(130 \, \text{m}^3/\text{min}\) in \(\text{m}^3/\text{h}\).
    \item (i) Convertire \(650 \, \text{cm/s}\) in \(\text{m/min}\).
    \item (j) Convertire \(120 \, \text{m/h}\) in \(\text{km/s}\).
    \item (k) Convertire \(36 \, \text{L/h}\) in \(\text{m}^3/\text{s}\).
    \item (l) Convertire \(100 \, \text{m}^3/\text{s}\) in \(\text{L/min}\).
\end{itemize}

\subsection*{Esercizio 3 (Notazione Scientifica)}
Convertire i seguenti numeri in notazione scientifica o viceversa:
\begin{itemize}
    \item (a) \(156\)
    \item (b) \(0.076\)
    \item (c) \(1879\)
    \item (d) \(0.001\)
    \item (e) \(23000000\)
    \item (f) \(0.00043\)
    \item (g) \(15000000000\)
    \item (h) \(0.000000000012\)
    \item (i) \(1.3 \times 10^7\)
    \item (j) \(4.2 \times 10^{-4}\)
    \item (k) \(9.1 \times 10^3\)
    \item (l) \(5.1 \times 10^{-2}\)
\end{itemize}

\subsection*{Esercizio 4 (Potenze di 10)}
Si eseguano i seguenti calcoli sfruttando il più possibile le regole sulle potenze di 10:
\begin{itemize}
    \item (a) \(10^8 \cdot 10^3\)
    \item (b) \(10^4 \cdot 10^{-2}\)
    \item (c) \(10^{-6} \cdot 10^8\)
    \item (d) \(10^{-1} \cdot 10^{-3}\)
    \item (e) \(10^8 : 10^3\)
    \item (f) \(10^4 : 10^{-2}\)
    \item (g) \(10^{-6} : 10^8\)
    \item (h) \(10^{-1} : 10^{-3}\)
\end{itemize}

\section*{Problemi}

\subsection*{Problemi sulla Densità}
\begin{itemize}
    \item \textbf{Esercizio 7:} La massa e il raggio di Giove sono rispettivamente \(m = 1.9 \times 10^{27} \, \text{kg}\) e \(r = 7.14 \times 10^4 \, \text{km}\). Si calcoli la densità di Giove in \(\text{kg/m}^3\).
    \item \textbf{Esercizio 8:} Il lato di un cubo di rame è \(\ell = 4.9 \, \text{cm}\).
        \begin{itemize}
            \item (a) Si calcoli il volume in \(\text{m}^3\) e in \(\text{L}\).
            \item (b) Sapendo che la densità del rame è \(d = 8960 \, \text{kg/m}^3\), calcolare la massa.
        \end{itemize}
\end{itemize}

\end{document}