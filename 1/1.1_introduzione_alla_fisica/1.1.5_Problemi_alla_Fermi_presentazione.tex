\documentclass{beamer}

\usetheme{CambridgeUS}
\usepackage{amsmath}

\title{Risoluzione di Problemi alla Fermi}
\author{Simulazione di Enrico Fermi}
\date{\today}

\begin{document}

\begin{frame}
    \titlepage
\end{frame}

% Introduzione
\begin{frame}{Introduzione}
    \begin{itemize}
        \item I problemi alla Fermi richiedono stime approssimative basate su dati incompleti. \pause
        \item L'obiettivo è ottenere una risposta con un ordine di grandezza ragionevole. \pause
        \item Metodo: \pause
        \begin{enumerate}
            \item Suddividere il problema in fattori. \pause
            \item Fare ipotesi ragionevoli. \pause
            \item Calcolare la stima.
        \end{enumerate}
    \end{itemize}
\end{frame}

% Problema 1
\begin{frame}{Problema 1: Accordatori di Pianoforte in Trentino}
    \textbf{Domanda:} Quanti accordatori di pianoforte ci sono in Trentino? \\
    \vspace{1em}
    \textbf{Dati e Assunzioni}: \pause
    \begin{itemize}
        \item Popolazione del Trentino \pause
        $5 \times 10^5$ persone. \pause
        \item Famiglia media  \pause
        2 persone. \pause
        \item Percentuale di famiglie con pianoforte  \pause 10\%. \pause
        \item Un accordatore lavora su \pause 1.000 pianoforti/anno. \pause
    \end{itemize}
    \pause
    \textbf{Calcolo}:
    \[
    \text{Pianoforti} = \frac{5 \cdot 10^5}{2} \times 0.1 = 10^4
    \]
    \pause
    \[
    \text{Accordatori} = \frac{10^4}{10^3} = 10^1
    \]
    \pause
    \textbf{Risultato:} circa 10 accordatori.
\end{frame}

% Problema 2
\begin{frame}{Problema 2: Consumo di Carburante in Italia}
    \textbf{Domanda:} Quanto carburante consumano le auto in un anno in un paese di 60 milioni di abitanti? \\
    \vspace{1em}
    \textbf{Soluzione 1}:
    
    \begin{itemize}
        \item In Italia ci sono 681 auto ogni 1.000 abitanti, cioè circa 1 ogni 2 abitanti.
        \item Numero di Auto Auto: $25 \times 10^6$. \pause
        \item Km percorsi/anno: $10^4$ km. \pause
        \item Consumo/km: $0.1$ litri.
    \end{itemize}
    \pause
    \textbf{Calcolo}:
    \[
    \text{Consumo totale} = 25 \times 10^6 \times 10^4 \times 0.1 = 2.5 \times 10^{10} \text{ litri.}
    \]
    \pause
    \textbf{Risultato:} $2.5 \times 10^{10}$ litri/anno.
\end{frame}

% Problema 3
\begin{frame}{Problema 3: Raggio della Terra}
    \textbf{Domanda:} Qual è il raggio terrestre? \\
    \vspace{1em}
    \textbf{Dati e Assunzioni}:
    \begin{itemize}
        \item Distanza tra fusi orari: 1.000 miglia. \pause
        \item Totale fusi orari: 24. \pause
        \item Circonferenza terrestre: $24 \times 10^3$ miglia.
    \end{itemize}
    \pause
    \textbf{Calcolo}:
    \[
    r = \frac{\text{Circonferenza}}{2\pi} = \frac{24 \times 10^3}{2 \pi} \approx 4 \times 10^3 \text{ miglia}.
    \]
    \pause
    \textbf{Risultato:} circa 6.400 km.
\end{frame}

% Problema 4
\begin{frame}{Problema 4: Consumo di Tè in Cina}
    \textbf{Domanda:} Quanto tè c'è in Cina? \\
    \vspace{1em}
    \textbf{Dati e Assunzioni}:
    \begin{itemize}
        \item Popolazione: $10^9$ persone. \pause
        \item Consumo giornaliero/persona: 10 g di foglie. \pause
        \item Durata delle scorte: $10^2$ giorni.
    \end{itemize}
    \pause
    \textbf{Calcolo}:
    \[
    \text{Tè totale} = 10^9 \times 10^{-2} \times 10^2 = 10^9 \text{ kg.}
    \]
    \pause
    \textbf{Risultato:} $10^9$ kg.
\end{frame}

% Problema 5
\begin{frame}{Problema 5: Numero di Respiri al Giorno}
    \textbf{Domanda:} Quanti respiri fa una persona in una giornata? \\
    \vspace{1em}
    \textbf{Dati e Assunzioni}:
    \begin{itemize}
        \item Respiri/minuto: 10. \pause
        \item Minuti in un giorno: $24 \times 60 = 1440$.
    \end{itemize}
    \pause
    \textbf{Calcolo}:
    \[
    \text{Respiri/giorno} = 10 \times 1440 \approx 1.4 \times 10^4.
    \]
    \pause
    \textbf{Risultato:} $10^4$ respiri/giorno.
\end{frame}

% Problema 6
\begin{frame}{Problema 6: Numero di Capelli}
    \textbf{Domanda:} Quanti capelli ha in media un essere umano? \\
    \vspace{1em}
    \textbf{Dati e Assunzioni}:
    \begin{itemize}
        \item Densità: $10^2$ capelli/cm². \pause
        \item Area del cuoio capelluto: $600 \, \text{cm}^2$.
    \end{itemize}
    \pause
    \textbf{Calcolo}:
    \[
    \text{Capelli totali} = 10^2 \times 600 = 10^5.
    \]
    \pause
    \textbf{Risultato:} $10^5$ capelli.
\end{frame}

% Conclusione
\begin{frame}{Conclusione}
    \begin{itemize}
        \item I problemi alla Fermi permettono di stimare ordini di grandezza con informazioni limitate. \pause
        \item Fondamentali in fisica, economia e innovazione. \pause
        \item Esercitandosi, è possibile migliorare la capacità di stima e il pensiero critico.
    \end{itemize}
\end{frame}

\end{document}