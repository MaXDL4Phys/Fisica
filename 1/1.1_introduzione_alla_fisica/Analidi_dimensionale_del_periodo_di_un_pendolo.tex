\documentclass{beamer}
\usetheme{Madrid}
\usecolortheme{seahorse} % Tema colori

% Pacchetti aggiuntivi
\usepackage{xcolor}

% Colori personalizzati
\setbeamercolor{title}{fg=teal}
\setbeamercolor{frametitle}{bg=teal!40, fg=white}
\setbeamercolor{structure}{fg=teal}


% Definizione dei colori personalizzati per teoria e problemi
\definecolor{theorycolor}{RGB}{173, 216, 230} % Blu chiaro per la teoria
\definecolor{problemcolor}{RGB}{255, 100, 0}   % Arancione per i problemi
\title{Analisi Dimensionale del Periodo del Pendolo}
\subtitle{}
\author{Prof. Massimo Bosetti}
\institute{Liceo da Vinci}
\date{}

\begin{document}

\begin{frame}
    \titlepage
\end{frame}
\setbeamercolor{frametitle}{bg=teal} % Colore per la teoria
\section{Introduzione}

\begin{frame}{Introduzione al Problema}
    \begin{itemize}
        \item Il periodo $ T $ di un pendolo semplice è una funzione della lunghezza $ L $, dell'accelerazione gravitazionale $ g $ e, potenzialmente, della massa $ M $.
        \item Scopo: Determinare, tramite l'analisi dimensionale, gli esponenti nella formula generica
        $$ T = k \cdot M^\gamma \cdot L^\alpha \cdot g^\beta $$
        dove $ k $ è una costante adimensionale.
    \end{itemize}
\end{frame}

\section{Definizione delle Dimensioni}

\begin{frame}{Definizione delle Dimensioni}
    \begin{itemize}
        \item $ T $: Periodo, con dimensioni di tempo $[T]$.
        \item $ M $: Massa del pendolo, con dimensioni di massa $[M]$.
        \item $ L $: Lunghezza del pendolo, con dimensioni di lunghezza $[L]$.
        \item $ g $: Accelerazione gravitazionale, con dimensioni di lunghezza diviso tempo al quadrato $[L \cdot T^{-2}]$.
    \end{itemize}
    \centering
    Scopo: trovare $\alpha$, $\beta$ e $\gamma$ affinché l'espressione $ T = M^\gamma L^\alpha g^\beta $ abbia dimensioni di tempo.
\end{frame}

\section{Costruzione della Formula Dimensionale}

\begin{frame}{Formula Dimensionale}
    Scriviamo le dimensioni dell'espressione:
    $$ [T] = [M]^\gamma \cdot [L]^\alpha \cdot [L \cdot T^{-2}]^\beta $$
    \begin{itemize}
        \item Espandendo le dimensioni di ciascun termine, otteniamo:
        $$ [T] = [M]^\gamma \cdot [L]^{\alpha + \beta} \cdot [T]^{-2\beta} $$
    \item Per determinare $\gamma$, $\alpha$ e $\beta$, uguagliamo le dimensioni di $ [M] $, $ [L] $, e $ [T] $.
    \end{itemize}
\end{frame}

\section{Equazioni Dimensionali}

\begin{frame}{Equazioni Dimensionali}
    Per ottenere le condizioni sugli esponenti, imponiamo che le dimensioni coincidano:
    \begin{itemize}
        \item Per la dimensione di massa $[M]$:
        $$ \gamma = 0 $$
        \item Per la dimensione di lunghezza $[L]$:
        $$ \alpha + \beta = 0 $$
        \item Per la dimensione di tempo $[T]$:
        $$ -2\beta = 1 $$
    \end{itemize}
\end{frame}

\section{Soluzione del Sistema}

\begin{frame}{Soluzione del Sistema di Equazioni}
    Risolviamo il sistema di equazioni:
    \begin{itemize}
        \item Dall'equazione $\gamma = 0$, la massa $ M $ non influenza il periodo $ T $.
        \item Dall'equazione $-2\beta = 1$, otteniamo:
        $$ \beta = -\frac{1}{2} $$
        \item Sostituendo $\beta = -\frac{1}{2}$ nell'equazione $\alpha + \beta = 0$:
        $$ \alpha - \frac{1}{2} = 0 \Rightarrow \alpha = \frac{1}{2} $$
    \end{itemize}
\end{frame}

\section{Risultato Finale}

\begin{frame}{Risultato Finale}
    Gli esponenti trovati sono:
    $$ \gamma = 0, \quad \alpha = \frac{1}{2}, \quad \beta = -\frac{1}{2} $$
    \begin{itemize}
        \item La formula dimensionale per il periodo $ T $ del pendolo è quindi:
        $$ T \propto L^{\frac{1}{2}} \cdot g^{-\frac{1}{2}} $$
        \item In termini di una costante adimensionale $ k $:
        $$ T = k \cdot \sqrt{\frac{L}{g}} $$
        con $ k = 2\pi $ nella formula esatta del periodo di un pendolo semplice.
    \end{itemize}
\end{frame}

\section{Conclusioni}

\begin{frame}{Conclusioni}
    \begin{itemize}
        \item L'analisi dimensionale mostra che il periodo di un pendolo dipende solo dalla lunghezza $ L $ e dall'accelerazione gravitazionale $ g $, ma non dalla massa $ M $.
        \item Abbiamo derivato la dipendenza del periodo: $ T \propto \sqrt{\frac{L}{g}} $.
        \item Questo risultato è coerente con la formula esatta per il periodo di un pendolo semplice.
    \end{itemize}
\end{frame}

\end{document}