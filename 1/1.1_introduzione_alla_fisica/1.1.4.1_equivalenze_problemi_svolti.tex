\documentclass{beamer}
\usetheme{Madrid}
\usecolortheme{seahorse} % Tema colori

% Pacchetti aggiuntivi
\usepackage{xcolor}

% Colori personalizzati
\setbeamercolor{title}{fg=blue}
\setbeamercolor{frametitle}{bg=blue!40, fg=white}
\setbeamercolor{structure}{fg=blue}


% Definizione dei colori personalizzati per teoria e problemi
\definecolor{theorycolor}{RGB}{173, 216, 230} % Blu chiaro per la teoria
\definecolor{problemcolor}{RGB}{255, 100, 0}   % Arancione per i problemi
\title{Equivalenze}
\subtitle{}
\author{Prof. Massimo Bosetti}
\institute{Liceo da Vinci}
\date{}

\begin{document}

\begin{frame}[fragile]
\frametitle{E1-Equivalenze - Esercizio 1}
Si eseguano le seguenti conversioni di unità di misura, specificando quale grandezza fisica viene espressa da ciascuna misura.

(i) Convertire 38700 mA in A.  
(ii) Convertire 0.001302 ms in µs.  
(iii) Convertire 37590 s in h.  
(iv) Convertire 830.3 dam in km.  
(v) Convertire 130 km/h in m/s.  
(vi) Convertire 7.8 g/cm\(^3\) in kg/m\(^3\).

\textbf{Soluzione:}  
(i) \(38700 \, \text{mA} = 38.7 \, \text{A}\). Intensità di corrente elettrica.  
(ii) \(0.001302 \, \text{ms} = 1.302 \, \mu\text{s}\). Intervallo di tempo.  
(iii) \(37590 \, \text{s} = 10.4 \, \text{h}\). Intervallo di tempo.  
(iv) \(830.3 \, \text{dam} = 8.033 \, \text{km}\). Lunghezza.  
(v) \(130 \, \text{km/h} = 36.1 \, \text{m/s}\). Velocità.  
(vi) \(7.8 \, \text{g/cm}^3 = 7800 \, \text{kg/m}^3\). Densità.
\end{frame}

\begin{frame}[fragile]
\frametitle{E1-Equivalenze - Esercizio 2}
Si eseguano le seguenti conversioni di unità di misura, avendo cura di esprimere il risultato in notazione scientifica.

(i) Convertire 384000 km in m.  
(ii) Convertire \(2.3 \times 10^4 \, \text{s}\) in ms.  
(iii) Convertire \(41.3 \, \text{hm}^2\) in \( \text{Gm}^2\).

\textbf{Soluzione:}  
(i) \(384000 \, \text{km} = 3.844 \times 10^8 \, \text{m}\).  
(ii) \(2.3 \times 10^4 \, \text{s} = 2.3 \times 10^7 \, \text{ms}\).  
(iii) \(41.3 \, \text{hm}^2 = 4.13 \times 10^{-13} \, \text{Gm}^2\).
\end{frame}

\begin{frame}[fragile]
\frametitle{E1-Equivalenze - Esercizio 3}
L’età della Terra è di circa \(4.6 \times 10^9\) anni. Si esprima l’età della Terra in secondi, scrivendo il risultato per mezzo della notazione scientifica.

\textbf{Soluzione:}  
\(1 \, \text{y} = 31536000 \, \text{s}\).  
\(4.6 \times 10^9 \, \text{y} = 4.6 \times 10^9 \times 31536000 \, \text{s} = 1.45 \times 10^{17} \, \text{s}\).
\end{frame}

\begin{frame}[fragile]
\frametitle{E1-Equivalenze - Esercizio 4}
Il diametro di una sfera è \(d = 4.2 \, \text{cm}\). Si calcoli il volume della sfera e si esprima il risultato in \( \text{m}^3\) e in litri.

\textbf{Soluzione:}  
Il raggio è \(r = d / 2 = 2.1 \, \text{cm}\).  
Il volume è \(V = \frac{4}{3} \pi r^3 = \frac{4}{3} \pi (2.1)^3 = 38.8 \, \text{cm}^3\).  
Convertendo in \( \text{m}^3\): \(V = 38.8 \times 10^{-6} \, \text{m}^3\).  
In litri: \(V = 0.0388 \, \text{L}\).
\end{frame}

\begin{frame}[fragile]
\frametitle{E1-Equivalenze - Esercizio 5}
La luce violetta di un arcobaleno ha una lunghezza d’onda \(\lambda = 435 \, \text{nm}\).

(i) Si esprima \(\lambda\) in metri.  
(ii) Quante lunghezze d’onda sono contenute in \(1 \, \text{m}\)?

\textbf{Soluzione:}  
(i) \(\lambda = 435 \, \text{nm} = 4.35 \times 10^{-7} \, \text{m}\).  
(ii) Numero di lunghezze d’onda in \(1 \, \text{m}\): \(1 \, \text{m} / 4.35 \times 10^{-7} \, \text{m} = 2298850\).
\end{frame}

\begin{frame}[fragile]
\frametitle{E1-Equivalenze - Esercizio 6}
La massa e il raggio della Terra sono rispettivamente \(m = 6 \times 10^{24} \, \text{kg}\) e \(r = 6378 \, \text{km}\). Si calcoli la densità della Terra, esprimendola in \( \text{kg/m}^3\).

\textbf{Soluzione:}  
Il volume è \(V = \frac{4}{3} \pi r^3 = \frac{4}{3} \pi (6378000)^3 \, \text{m}^3 = 1.09 \times 10^{21} \, \text{m}^3\).  
La densità è \(d = m / V = 6 \times 10^{24} / 1.09 \times 10^{21} = 5521 \, \text{kg/m}^3\).
\end{frame}

\end{document}