\documentclass{beamer}
\usetheme{Madrid}
\usecolortheme{seahorse} % Tema colori

% Pacchetti aggiuntivi
\usepackage{xcolor}

% Colori personalizzati
\setbeamercolor{title}{fg=cyan}
\setbeamercolor{frametitle}{bg=blue!40, fg=white}
\setbeamercolor{structure}{fg=cyan}

\title{I Vettori in 3D}
\subtitle{Concetti fondamentali e operazioni}
\author{}
\institute{}
\date{}

\begin{document}

\begin{frame}
    \titlepage
\end{frame}

\section{Rappresentazione dei Vettori}

\begin{frame}{Rappresentazione dei Vettori}
    \begin{itemize}
        \item Un vettore in 3D è rappresentato da una terna di componenti \((x, y, z)\) che descrivono la sua posizione rispetto agli assi.
        \item Viene solitamente rappresentato con un segmento orientato in uno spazio tridimensionale.
        \item La lunghezza del vettore rappresenta la sua magnitudine, mentre la direzione del segmento rappresenta la direzione del vettore.
    \end{itemize}
\end{frame}

\begin{frame}{Esempio Semplice: Rappresentazione di un Vettore}
    \begin{itemize}
        \item Consideriamo il vettore \(\vec{v} = (2, 3, 4)\).
        \item Questo vettore ha componenti lungo gli assi \(x\), \(y\) e \(z\).
        \item La magnitudine del vettore è:
        \[
        |\vec{v}| = \sqrt{2^2 + 3^2 + 4^2} = \sqrt{29}
        \]
    \end{itemize}
\end{frame}

\begin{frame}{Applicazione Complessa: Rappresentazione in Coordinate Polari}
    \begin{itemize}
        \item Convertire il vettore \(\vec{v} = (1, 1, 1)\) in coordinate sferiche.
        \item Calcolare la magnitudine:
        \[
        r = \sqrt{1^2 + 1^2 + 1^2} = \sqrt{3}
        \]
        \item Calcolare gli angoli:
        \[
        \theta = \arccos\left(\frac{z}{r}\right), \quad \phi = \arctan\left(\frac{y}{x}\right)
        \]
        \item Quindi, le coordinate sferiche sono \((r, \theta, \phi)\).
    \end{itemize}
\end{frame}

\section{Operazioni sui Vettori}

\begin{frame}{Somma di Vettori}
    \begin{itemize}
        \item La somma di due vettori \( \vec{A} \) e \( \vec{B} \) si ottiene sommando le componenti corrispondenti:
        \[
        \vec{A} + \vec{B} = (A_x + B_x, A_y + B_y, A_z + B_z)
        \]
        \item Graficamente, la somma dei vettori è rappresentata dal "metodo del parallelogramma" o dalla "regola del triangolo".
    \end{itemize}
\end{frame}

\begin{frame}{Esercizio Semplice: Somma di Vettori}
    \begin{itemize}
        \item Dati i vettori \( \vec{A} = (1, 2, 3) \) e \( \vec{B} = (4, -1, 2) \).
        \item Calcolare \( \vec{A} + \vec{B} \).
        \item Soluzione:
        \[
        \vec{A} + \vec{B} = (1+4, 2+(-1), 3+2) = (5, 1, 5)
        \]
    \end{itemize}
\end{frame}

\begin{frame}{Applicazione Complessa: Somma di Forze}
    \begin{itemize}
        \item Due forze \( \vec{F_1} \) e \( \vec{F_2} \) agiscono su un corpo.
        \item \( \vec{F_1} = (10, 0, 0) \) N, \( \vec{F_2} = (0, 5, 0) \) N.
        \item Calcolare la forza risultante \( \vec{F_R} \) e la sua magnitudine.
        \item Soluzione:
        \[
        \vec{F_R} = \vec{F_1} + \vec{F_2} = (10, 5, 0)
        \]
        \[
        |\vec{F_R}| = \sqrt{10^2 + 5^2 + 0^2} = \sqrt{125} = 11,18 \text{ N}
        \]
    \end{itemize}
\end{frame}

\begin{frame}{Differenza di Vettori}
    \begin{itemize}
        \item La differenza tra due vettori \( \vec{A} \) e \( \vec{B} \) si ottiene sottraendo le componenti corrispondenti:
        \[
        \vec{A} - \vec{B} = (A_x - B_x, A_y - B_y, A_z - B_z)
        \]
        \item La differenza di vettori può essere vista come la somma del primo vettore con l'opposto del secondo.
    \end{itemize}
\end{frame}

\begin{frame}{Esercizio Semplice: Differenza di Vettori}
    \begin{itemize}
        \item Dati \( \vec{A} = (5, 4, 3) \) e \( \vec{B} = (2, 1, 0) \).
        \item Calcolare \( \vec{A} - \vec{B} \).
        \item Soluzione:
        \[
        \vec{A} - \vec{B} = (5-2, 4-1, 3-0) = (3, 3, 3)
        \]
    \end{itemize}
\end{frame}

\begin{frame}{Applicazione Complessa: Spostamento Netto}
    \begin{itemize}
        \item Un oggetto si sposta dal punto \( P_1 = (2, 3, 5) \) al punto \( P_2 = (7, 6, 9) \).
        \item Calcolare il vettore spostamento \( \Delta\vec{D} \).
        \item Soluzione:
        \[
        \Delta\vec{D} = P_2 - P_1 = (7-2, 6-3, 9-5) = (5, 3, 4)
        \]
        \item La distanza percorsa è:
        \[
        |\Delta\vec{P}| = \sqrt{5^2 + 3^2 + 4^2} = \sqrt{50} = 7,07
        \]
    \end{itemize}
\end{frame}

\begin{frame}{Moltiplicazione per uno Scalare}
    \begin{itemize}
        \item Dato un vettore \( \vec{A} = (A_x, A_y, A_z) \) e uno scalare \( k \), la moltiplicazione per uno scalare modifica la lunghezza del vettore:
        \[
        k \cdot \vec{A} = (k \cdot A_x, k \cdot A_y, k \cdot A_z)
        \]
        \item Se \( k = 1 \), il vettore rimane invariato. Se \( k = -1 \), otteniamo il vettore opposto.
        \item Per \( |k| < 1 \), il vettore si riduce in lunghezza; per \( |k| > 1 \), il vettore si allunga.
    \end{itemize}
\end{frame}

\begin{frame}{Esercizio Semplice: Moltiplicazione per uno Scalare}
    \begin{itemize}
        \item Dato \( \vec{A} = (3, -2, 1) \) e \( k = 2 \).
        \item Calcolare \( k \cdot \vec{A} \).
        \item Soluzione:
        \[
        2 \cdot \vec{A} = (2 \cdot 3, 2 \cdot -2, 2 \cdot 1) = (6, -4, 2)
        \]
    \end{itemize}
\end{frame}

\begin{frame}{Applicazione Complessa: Vettore Velocità}
    \begin{itemize}
        \item Un oggetto si muove con velocità \( \vec{v} = (5, 0, 0) \) m/s.
        \item Dopo 3 secondi, calcolare lo spostamento \( \vec{s} \).
        \item Soluzione:
        \[
        \vec{s} = \vec{v} \cdot t = (5 \cdot 3, 0, 0) = (15, 0, 0) \text{ m}
        \]
    \end{itemize}
\end{frame}

\section{Prodotti tra Vettori}

\begin{frame}{Prodotto Scalare}
    \begin{itemize}
        \item Il prodotto scalare tra due vettori \( \vec{A} \) e \( \vec{B} \) restituisce uno scalare:
        \[
        \vec{A} \cdot \vec{B} = A_x B_x + A_y B_y + A_z B_z
        \]
        \item Questo prodotto misura la "proiezione" di un vettore sull'altro e indica la loro perpendicolarità:
        \[
        \vec{A} \cdot \vec{B} = 0 \Rightarrow \text{vettori perpendicolari}
        \]
    \end{itemize}
\end{frame}

\begin{frame}{Esercizio Semplice: Calcolo del Prodotto Scalare}
    \begin{itemize}
        \item Dati \( \vec{A} = (1, 0, 0) \) e \( \vec{B} = (0, 1, 0) \).
        \item Calcolare \( \vec{A} \cdot \vec{B} \).
        \item Soluzione:
        \[
        \vec{A} \cdot \vec{B} = (1 \cdot 0) + (0 \cdot 1) + (0 \cdot 0) = 0
        \]
        \item I vettori sono perpendicolari.
    \end{itemize}
\end{frame}

\begin{frame}{Applicazione Complessa: Angolo tra Due Vettori}
    \begin{itemize}
        \item Dati \( \vec{A} = (3, -3, 1) \) e \( \vec{B} = (4, 9, 2) \).
        \item Calcolare l'angolo \( \theta \) tra \( \vec{A} \) e \( \vec{B} \).
        \item Soluzione:
        \[
        \vec{A} \cdot \vec{B} = (3 \cdot 4) + (-3 \cdot 9) + (1 \cdot 2) = 12 -27 + 2 = -13
        \]
        \[
        |\vec{A}| = \sqrt{3^2 + (-3)^2 + 1^2} = \sqrt{19}
        \]
        \[
        |\vec{B}| = \sqrt{4^2 + 9^2 + 2^2} = \sqrt{101}
        \]
        \[
        \cos\theta = \frac{\vec{A} \cdot \vec{B}}{|\vec{A}||\vec{B}|} = \frac{-13}{\sqrt{19} \sqrt{101}}
        \]
        \[
        \theta = \arccos\left(\frac{-13}{\sqrt{19} \sqrt{101}}\right)
        \]
    \end{itemize}
\end{frame}

% \begin{frame}{Prodotto Vettoriale}
%     \begin{itemize}
%         \item Il prodotto vettoriale tra due vettori \( \vec{A} \) e \( \vec{B} \) genera un nuovo vettore \( \vec{C} \), ortogonale sia a \( \vec{A} \) che a \( \vec{B} \).
%         \[
%         \vec{A} \times \vec{B} = (A_y B_z - A_z B_y, A_z B_x - A_x B_z, A_x B_y - A_y B_x)
%         \]
%         \item La direzione del vettore risultante segue la regola della mano destra.
%         \item La magnitudine di \( \vec{C} \) è data da \( |\vec{C}| = |\vec{A}||\vec{B}| \sin(\theta) \), dove \( \theta \) è l'angolo tra \( \vec{A} \) e \( \vec{B} \).
%     \end{itemize}
% \end{frame}

\begin{frame}{Prodotto Vettoriale}
    \begin{itemize}
 
        \item Il prodotto vettoriale può essere calcolato come il determinante di una matrice:
        \begin{equation*}
        \vec{C} =\vec{A} \times \vec{B} = \begin{vmatrix}
        \mathbf{\hat{i}} & \mathbf{\hat{j}} & \mathbf{\hat{k}} \\
        A_x & A_y & A_z \\
        B_x & B_y & B_z \\
        \end{vmatrix}\end{equation*}
        $$ \vec{C} =\vec{A} \times \vec{B} =  \mathbf{\hat{i}} (A_y B_z - A_z B_y) +  \mathbf{\hat{j}}(A_z B_x - A_x B_z) +  \mathbf{\hat{k}}(A_x B_y - A_y B_x)
        $$
        \item Il prodotto vettoriale tra due vettori \( \vec{A} \) e \( \vec{B} \) genera un nuovo vettore \( \vec{C} \), ortogonale sia a \( \vec{A} \) che a \( \vec{B} \).
        \item La direzione del vettore risultante segue la regola della mano destra.
        \item Il modulo di \( \vec{C} \) è data da \( |\vec{C}| = |\vec{A}||\vec{B}| \sin(\theta) \), dove \( \theta \) è l'angolo tra \( \vec{A} \) e \( \vec{B} \).
    \end{itemize}
\end{frame}

\begin{frame}{Esercizio Semplice: Calcolo del Prodotto Vettoriale}
    \begin{itemize}
        \item Dati \( \vec{A} = (1, 0, 0) \) e \( \vec{B} = (0, 1, 0) \).
        \item Calcolare \( \vec{A} \times \vec{B} \).
        \item Soluzione:
        \[
        \vec{A} \times \vec{B} = (0 \cdot 0 - 0 \cdot 1, 0 \cdot 0 - 1 \cdot 0, 1 \cdot 1 - 0 \cdot 0) = (0, 0, 1)
        \]
        \item Il risultato è il vettore unitario lungo l'asse \( z \).
    \end{itemize}
\end{frame}

\begin{frame}{Applicazione Complessa: Momento di una Forza}
    \begin{itemize}
        \item Una forza \( \vec{F} = (5, 0, 0) \) N viene applicata nel punto \( \vec{r} = (0, 2, 0) \) m.
        \item Calcolare il momento \( \vec{M} = \vec{r} \times \vec{F} \).
        \item Soluzione:
        \[
        \vec{M} = (2 \cdot 0 - 0 \cdot 0, 0 \cdot 5 - 0 \cdot 0, 0 \cdot 0 - 2 \cdot 5) = (0, 0, -10)
        \]
        \item Il momento è \( \vec{M} = (0, 0, -10) \) N·m, indicando una rotazione attorno all'asse \( z \).
    \end{itemize}
\end{frame}

\end{document}