\documentclass[a4paper,12pt]{article}
\usepackage[utf8]{inputenc}
\usepackage[italian]{babel}
\usepackage{amsmath, amssymb}
\usepackage{graphicx}
\usepackage{tikz}
\usepackage{geometry}
\geometry{margin=2.5cm}

\title{Fisica: Dinamica, Forze ed Equilibrio}
\author{A cura di Richard Feynman}
\date{}

\begin{document}

\maketitle
\tableofcontents
\newpage

\section{Introduzione}

Il primo principio della dinamica afferma che un corpo rimane nel suo stato di quiete o di moto rettilineo uniforme se la risultante delle forze che agiscono su di esso è nulla. Questo principio, noto anche come principio di inerzia, è fondamentale per comprendere il comportamento dei corpi in equilibrio.

\section{Forze e Equilibrio}

Un oggetto è in equilibrio quando la somma vettoriale di tutte le forze agenti su di esso è zero. Le principali forze da considerare sono:

\subsection{Forza Peso}

La forza peso $\vec{P}$ è la forza con cui un oggetto è attratto dalla Terra, data da:
\[
\vec{P} = m \cdot \vec{g}
\]
dove $m$ è la massa dell'oggetto e $\vec{g}$ è l'accelerazione di gravità.

\subsection{Forza Normale}

La forza normale $\vec{N}$ è la reazione della superficie su cui si trova l'oggetto ed è perpendicolare ad essa.

\subsection{Tensione}

La tensione $\vec{T}$ è la forza trasmessa da una corda, cavo o filo quando viene tirato. Si assume che la corda sia inestensibile e di massa trascurabile.

\subsection{Forza Trainante}

La forza trainante è la forza applicata per tirare o spingere un oggetto.

\section{Scomposizione delle Forze}

Per analizzare l'equilibrio di un corpo, è spesso necessario scomporre le forze nelle loro componenti lungo assi specifici.

\begin{figure}[h]
    \centering
    % Esempio di scomposizione delle forze su un piano inclinato
    \begin{tikzpicture}
        \draw (0,0) -- (5,0);
        \draw (0,0) -- (5,2);
        \draw (5,0) -- (5,2);
        \draw[fill] (2.5,1) rectangle (2.7,1.2);
        \draw[->] (2.6,1.1) -- (2.6,0) node[right] {$\vec{P}$};
        \draw[->] (2.6,1.1) -- (3.6,1.1) node[above] {$\vec{P}_\parallel$};
        \draw[->] (2.6,1.1) -- (2.6,2) node[left] {$\vec{N}$};
        \node at (0.8,0.2) {$\alpha$};
    \end{tikzpicture}
    \caption{Scomposizione delle forze su un piano inclinato}
\end{figure}

\section{Piano Inclinato}

Un piano inclinato è una superficie che forma un angolo $\alpha$ con l'orizzontale. Su un piano inclinato, la forza peso si scompone in:

\[
\vec{P}_\parallel = m \cdot g \cdot \sin{\alpha}
\]
\[
\vec{P}_\perp = m \cdot g \cdot \cos{\alpha}
\]

\section{Applicazione Pratica}

Un esempio pratico è il calcolo della reazione vincolare in un ponte, dove è necessario considerare le forze per garantire la stabilità della struttura.

\section{Esempi Svolti}

Di seguito sono presentati 10 esempi svolti sull'equilibrio delle forze.

\subsection*{Esempio 1}

\textbf{Problema:} Un blocco di massa $10\,\text{kg}$ è posizionato su un piano inclinato di $30^\circ$. Calcolare la componente parallela e perpendicolare della forza peso.

\textbf{Soluzione:}

Calcoliamo le componenti della forza peso:

\[
\vec{P}_\parallel = m \cdot g \cdot \sin{\alpha} = 10\,\text{kg} \times 9.81\,\text{m/s}^2 \times \sin{30^\circ} = 49.05\,\text{N}
\]

\[
\vec{P}_\perp = m \cdot g \cdot \cos{\alpha} = 10\,\text{kg} \times 9.81\,\text{m/s}^2 \times \cos{30^\circ} = 84.87\,\text{N}
\]

\begin{figure}[h]
    \centering
    \begin{tikzpicture}
        \draw (0,0) -- (5,0);
        \draw (0,0) -- (5,2.5);
        \draw (5,0) -- (5,2.5);
        \draw[fill] (2.5,1.25) rectangle (2.7,1.45);
        \draw[->] (2.6,1.35) -- (2.6,0) node[right] {$\vec{P}$};
        \draw[->] (2.6,1.35) -- (4,1.35) node[above] {$\vec{P}_\parallel$};
        \draw[->] (2.6,1.35) -- (2.6,2.5) node[left] {$\vec{N}$};
        \node at (0.8,0.2) {$30^\circ$};
    \end{tikzpicture}
    \caption{Blocco su piano inclinato di $30^\circ$}
\end{figure}

\subsection*{Esempio 2}

\textbf{Problema:} Un oggetto di massa $5\,\text{kg}$ è sospeso a due cavi che formano angoli di $45^\circ$ con l'orizzontale. Calcolare la tensione in ciascun cavo.

\textbf{Soluzione:}

La somma delle componenti verticali delle tensioni deve bilanciare la forza peso:

\[
2 T \sin{45^\circ} = m \cdot g
\]

\[
T = \frac{m \cdot g}{2 \sin{45^\circ}} = \frac{5\,\text{kg} \times 9.81\,\text{m/s}^2}{2 \times \frac{\sqrt{2}}{2}} = 34.7\,\text{N}
\]

\begin{figure}[h]
    \centering
    \begin{tikzpicture}
        \draw (0,0) -- (2,2);
        \draw (4,0) -- (2,2);
        \draw[fill] (2,2) circle [radius=0.1];
        \draw[fill] (2,0) rectangle (2.2,0.2);
        \draw[->] (2.1,2) -- (2.1,0.2) node[right] {$\vec{P}$};
        \node at (1,1) {$T$};
        \node at (3,1) {$T$};
    \end{tikzpicture}
    \caption{Oggetto sospeso a due cavi}
\end{figure}

\subsection*{Esempio 3}

\textbf{Problema:} Un blocco di massa $8\,\text{kg}$ è su un piano inclinato di $20^\circ$. Determinare la forza normale esercitata dal piano sul blocco.

\textbf{Soluzione:}

La forza normale è uguale alla componente perpendicolare della forza peso:

\[
\vec{N} = m \cdot g \cdot \cos{\alpha} = 8\,\text{kg} \times 9.81\,\text{m/s}^2 \times \cos{20^\circ} = 73.68\,\text{N}
\]

\begin{figure}[h]
    \centering
    \begin{tikzpicture}
        \draw (0,0) -- (5,0);
        \draw (0,0) -- (5,1.82);
        \draw (5,0) -- (5,1.82);
        \draw[fill] (2.5,0.91) rectangle (2.7,1.11);
        \draw[->] (2.6,1.01) -- (2.6,0) node[right] {$\vec{P}$};
        \draw[->] (2.6,1.01) -- (3.8,1.01) node[above] {$\vec{P}_\parallel$};
        \draw[->] (2.6,1.01) -- (2.6,1.82) node[left] {$\vec{N}$};
        \node at (0.8,0.2) {$20^\circ$};
    \end{tikzpicture}
    \caption{Blocco su piano inclinato di $20^\circ$}
\end{figure}

\subsection*{Esempio 4}

\textbf{Problema:} Un oggetto di massa $15\,\text{kg}$ è sospeso da un cavo verticale. Calcolare la tensione nel cavo.

\textbf{Soluzione:}

La tensione nel cavo bilancia la forza peso:

\[
T = m \cdot g = 15\,\text{kg} \times 9.81\,\text{m/s}^2 = 147.15\,\text{N}
\]

\subsection*{Esempio 5}

\textbf{Problema:} Un corpo di massa $12\,\text{kg}$ è trainato su un piano orizzontale con una forza di $30\,\text{N}$ applicata con un angolo di $45^\circ$ rispetto all'orizzontale. Determinare la reazione vincolare verticale.

\textbf{Soluzione:}

La componente verticale della forza trainante riduce la forza normale:

\[
F_{\text{verticale}} = F \cdot \sin{\theta} = 30\,\text{N} \times \sin{45^\circ} = 21.21\,\text{N}
\]

La forza normale è:

\[
\vec{N} = m \cdot g - F_{\text{verticale}} = 12\,\text{kg} \times 9.81\,\text{m/s}^2 - 21.21\,\text{N} = 96.51\,\text{N}
\]

\subsection*{Esempio 6}

\textbf{Problema:} Un piano inclinato ha un angolo di $60^\circ$. Qual è la forza normale su un blocco di $2\,\text{kg}$ posto su di esso?

\textbf{Soluzione:}

\[
\vec{N} = m \cdot g \cdot \cos{\alpha} = 2\,\text{kg} \times 9.81\,\text{m/s}^2 \times \cos{60^\circ} = 9.81\,\text{N}
\]

\subsection*{Esempio 7}

\textbf{Problema:} Un oggetto è in equilibrio sotto l'azione di tre forze. Due forze sono $\vec{F}_1 = 10\,\text{N}$ a $0^\circ$ e $\vec{F}_2 = 15\,\text{N}$ a $90^\circ$. Calcolare la terza forza $\vec{F}_3$ necessaria per mantenere l'equilibrio.

\textbf{Soluzione:}

La risultante delle prime due forze è:

\[
\vec{R} = \vec{F}_1 + \vec{F}_2
\]

Componenti:

\[
R_x = F_1 + 0 = 10\,\text{N}
\]
\[
R_y = 0 + F_2 = 15\,\text{N}
\]

La terza forza deve essere opposta a $\vec{R}$:

\[
\vec{F}_3 = -\vec{R}
\]

Modulo di $\vec{F}_3$:

\[
F_3 = \sqrt{R_x^2 + R_y^2} = \sqrt{10^2 + 15^2} = 18.03\,\text{N}
\]

Angolo di $\vec{F}_3$ rispetto all'asse $x$:

\[
\theta = \arctan{\left(\frac{R_y}{R_x}\right)} = \arctan{\left(\frac{15}{10}\right)} = 56.31^\circ
\]

Quindi, la terza forza è $18.03\,\text{N}$ a $236.31^\circ$ (poiché è opposta a $\vec{R}$).

\subsection*{Esempio 8}

\textbf{Problema:} Calcolare la tensione in due cavi che sostengono un carico di $20\,\text{kg}$, formando angoli di $30^\circ$ e $60^\circ$ con l'orizzontale.

\textbf{Soluzione:}

Le componenti verticali delle tensioni devono bilanciare la forza peso:

\[
T_1 \sin{30^\circ} + T_2 \sin{60^\circ} = m \cdot g = 196.2\,\text{N}
\]

Le componenti orizzontali devono annullarsi:

\[
T_1 \cos{30^\circ} = T_2 \cos{60^\circ}
\]

Risolvendo il sistema:

1. Dalla seconda equazione:

\[
T_1 = T_2 \frac{\cos{60^\circ}}{\cos{30^\circ}} = T_2 \left(\frac{0.5}{\sqrt{3}/2}\right) = T_2 \left(\frac{0.5}{0.866}\right) = T_2 \times 0.577
\]

2. Sostituendo nella prima equazione:

\[
0.577 T_2 \sin{30^\circ} + T_2 \sin{60^\circ} = 196.2\,\text{N}
\]

\[
0.577 T_2 \times 0.5 + T_2 \times \frac{\sqrt{3}}{2} = 196.2
\]

\[
0.2885 T_2 + 0.866 T_2 = 196.2
\]

\[
1.1545 T_2 = 196.2
\]

\[
T_2 = \frac{196.2}{1.1545} = 170\,\text{N}
\]

\[
T_1 = 0.577 \times 170 = 98\,\text{N}
\]

\subsection*{Esempio 9}

\textbf{Problema:} Un oggetto è spinto su per un piano inclinato senza attrito con una forza parallela al piano. Se la forza è di $50\,\text{N}$, l'oggetto ha massa $5\,\text{kg}$ e l'angolo del piano è $30^\circ$, qual è l'accelerazione dell'oggetto?

\textbf{Soluzione:}

La componente della forza peso lungo il piano è:

\[
\vec{P}_\parallel = m \cdot g \cdot \sin{\alpha} = 5\,\text{kg} \times 9.81\,\text{m/s}^2 \times \sin{30^\circ} = 24.525\,\text{N}
\]

La forza netta è:

\[
F_{\text{netta}} = F_{\text{applicata}} - \vec{P}_\parallel = 50\,\text{N} - 24.525\,\text{N} = 25.475\,\text{N}
\]

L'accelerazione è:

\[
a = \frac{F_{\text{netta}}}{m} = \frac{25.475\,\text{N}}{5\,\text{kg}} = 5.095\,\text{m/s}^2
\]

\subsection*{Esempio 10}

\textbf{Problema:} Determinare la forza necessaria per mantenere in equilibrio un blocco su un piano inclinato di $25^\circ$, considerando un attrito con coefficiente $\mu = 0.1$. Il blocco ha una massa di $10\,\text{kg}$.

\textbf{Soluzione:}

La forza di attrito è:

\[
F_{\text{attrito}} = \mu \cdot \vec{N}
\]

La forza normale è:

\[
\vec{N} = m \cdot g \cdot \cos{\alpha} = 10\,\text{kg} \times 9.81\,\text{m/s}^2 \times \cos{25^\circ} = 88.89\,\text{N}
\]

Quindi:

\[
F_{\text{attrito}} = 0.1 \times 88.89\,\text{N} = 8.889\,\text{N}
\]

La componente parallela della forza peso è:

\[
\vec{P}_\parallel = m \cdot g \cdot \sin{25^\circ} = 10\,\text{kg} \times 9.81\,\text{m/s}^2 \times \sin{25^\circ} = 41.42\,\text{N}
\]

La forza necessaria per mantenere l'equilibrio è:

\[
F = \vec{P}_\parallel - F_{\text{attrito}} = 41.42\,\text{N} - 8.889\,\text{N} = 32.53\,\text{N}
\]

\section{Esercizi da Svolgere}

\begin{enumerate}
    \item Un blocco di massa $8\,\text{kg}$ è su un piano inclinato di $20^\circ$. Calcolare $\vec{P}_\parallel$ e $\vec{P}_\perp$.
    \item Un oggetto di $15\,\text{kg}$ è sospeso a un cavo. Determinare la tensione nel cavo.
    \item Un piano inclinato ha un angolo di $60^\circ$. Qual è la forza normale su un blocco di $2\,\text{kg}$?
    \item Un oggetto è in equilibrio sotto l'azione di tre forze. Due forze sono di $10\,\text{N}$ e $15\,\text{N}$ con angoli di $0^\circ$ e $90^\circ$. Calcolare la terza forza.
    \item Un corpo di massa $12\,\text{kg}$ è trainato su un piano orizzontale con una forza di $30\,\text{N}$ applicata con un angolo di $45^\circ$ rispetto all'orizzontale. Determinare la reazione vincolare.
    \item Calcolare la tensione in due cavi che sostengono un carico di $20\,\text{kg}$, formando angoli di $30^\circ$ e $60^\circ$ con l'orizzontale.
    \item Un blocco è in equilibrio su un piano inclinato senza attrito. Se l'angolo del piano è aumentato, cosa succede alle componenti della forza peso?
    \item Un oggetto è spinto su per un piano inclinato con una forza parallela al piano. Se la forza è di $50\,\text{N}$ e l'oggetto ha massa $5\,\text{kg}$, qual è l'accelerazione?
    \item Determinare la forza necessaria per mantenere in equilibrio un blocco su un piano inclinato di $25^\circ$ considerando attrito con coefficiente $\mu = 0.1$.
    \item Un ponte sostiene un carico distribuito uniformemente di $1000\,\text{kg}$. Calcolare la reazione vincolare totale del ponte.
\end{enumerate}

\end{document}