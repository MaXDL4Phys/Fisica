\documentclass{beamer}
\usetheme{Madrid}
\usecolortheme{seahorse} % Tema colori

% Pacchetti aggiuntivi
\usepackage{xcolor}
\usepackage[utf8]{inputenc}
\usepackage[italian]{babel}
\usepackage{amsmath, amssymb}
\usepackage{graphicx}
\usepackage{tikz}
\usetikzlibrary{calc} % Libreria per il calcolo delle coordinate
\usetikzlibrary{scopes}

\title{Fisica: Dinamica, Forze ed Equilibrio}
\author{Prof. M Bosetti}
\date{}

\begin{document}

\frame{\titlepage}

\begin{frame}
\frametitle{Indice}
\tableofcontents
\end{frame}

\section{Introduzione}

\begin{frame}
\frametitle{Primo Principio della Dinamica}
Il primo principio della dinamica afferma che se la risultante delle forze agenti su un corpo è nulla, il corpo rimane in stato di quiete o di moto rettilineo uniforme.
\end{frame}

\section{Forze e Equilibrio}

\begin{frame}
\frametitle{Forza Peso}
\begin{equation*}
\vec{P} = m \cdot \vec{g}
\end{equation*}
La forza con cui un oggetto è attratto dalla Terra.
\end{frame}

\begin{frame}
\frametitle{Forza Normale e Tensione}
\begin{itemize}
    \item \textbf{Forza Normale} ($\vec{N}$): Reazione perpendicolare della superficie.
    \item \textbf{Tensione} ($\vec{T}$): Forza trasmessa da una corda o cavo.
\end{itemize}
\end{frame}

% \section{Scomposizione delle Forze}

\begin{frame}
\frametitle{Perché Scomporre le Forze?}
Per analizzare le forze agenti in diverse direzioni e risolvere problemi di equilibrio.
\end{frame}

\begin{frame}
\frametitle{Il Piano Inclianto: scomposizione della forz apeso}
\newcommand{\ang}{30}
\begin{tikzpicture} [font = \small]
% triangle:
\draw [draw = orange, fill = orange!15] (0,0) coordinate (O) -- (\ang:6)
	coordinate [pos=.45] (M) |- coordinate (B) (O);

% angles:
\draw [draw = orange] (O) ++(.8,0) arc (0:\ang:0.8) 
	node [pos=.4, left] {$\alpha$};
\draw [draw = orange] (B) rectangle ++(-0.3,0.3);

\begin{scope} [-latex,rotate=\ang]
% Object (rectangle)
\draw [fill = purple!30,
	draw = purple!50] (M) rectangle ++ (1,.6);

% Weight Force and its projections
\draw [dashed, red] (M) ++ (.5,.3) coordinate (MM) -- ++ (0,-1.29)
	node [very near end, right] {$\vec{P}_{\perp} $};

\draw [dashed, teal] (MM) -- ++ (-0.75,0) 
	node [very near end, left] {$\vec{P}_\parallel $};

\draw (MM) -- ++ (-\ang-90:1.5)
	node [very near end, below left ] {$mg$};

% Normal Force
\draw [blue] (MM) -- ++ (0,1.29)
node [very near end, right] {$N$};

% Frictional Force
% \draw (MM) -- ++ (0.75,0)
% 	node [very near end, above] {$f$};
\end{scope}
\end{tikzpicture} 

\begin{equation*}
 \left\{\begin{matrix}
\vec{P}_\parallel = m \cdot g \cdot \sin{\alpha} \\
\vec{P}_\perp = m \cdot g \cdot \cos{\alpha}
\end{matrix}\right.
\end{equation*}
\end{frame}

\section{Esempi Svolti}
\begin{frame}
\frametitle{Esempio 1}
\textbf{Problema:} Un blocco di massa $10\,\text{kg}$ è su un piano inclinato di $30^\circ$. Calcolare $\vec{P}_\parallel$ e $\vec{P}_\perp$.

\textbf{Soluzione:}
\begin{equation*}
\vec{P}_\parallel = 10\,\text{kg} \times 9.81\,\text{m/s}^2 \times \sin{30^\circ} = 49.05\,\text{N}
\end{equation*}
\begin{equation*}
\vec{P}_\perp = 10\,\text{kg} \times 9.81\,\text{m/s}^2 \times \cos{30^\circ} = 84.87\,\text{N}
\end{equation*}
\end{frame}

\begin{frame}
\frametitle{Esempio 2}
\textbf{Problema:} Un oggetto di massa $5\,\text{kg}$ è sospeso a due cavi che formano angoli di $45^\circ$ con l'orizzontale. Calcolare la tensione in ciascun cavo.

\textbf{Soluzione:}
\begin{figure}[h]
    \centering
\begin{tikzpicture}[font = \small]
  % Posizione del punto di sospensione e massa
  % \coordinate (O) at (0,0); % Punto di sospensione
  \coordinate (A) at (-1,-2); % Punto in cui termina il cavo sinistro
  \coordinate (B) at (1,-2); % Punto in cui termina il cavo destro
  \coordinate (M) at (0,-3); % Massa

  % Disegna i cavi (dalla massa verso l'alto)
  \draw[->, thick, red] (M) -- (A) node[  left] {$T_1$};
  \draw[->, thick, blue] (M) -- (B) node[  right] {$T_2$};

  % Disegna la massa
  \draw[fill=gray!50] (M) circle (0.2) ;

  %disegna il piano orizzonatle
  \draw[dashed, thick] (-2,-3) -- (2,-3);

  % Disegna la forza peso
  \draw[->, thick, black] (M) -- ++(0,-1) node[below] {$\vec{P} = m\cdot \vec{g} = 5\,\text{kg} \cdot 9.8\,\frac{\text{m}}{\text{s}^2} = 49\,\text{N}$};

  % Componenti delle tensioni
  \draw[->, dashed, red] (-1, -3) -- node[ midway, left ] {$T_{1y} = T_1 \sin\theta$} (A) ;
  \draw[dashed, blue] (1, -3) --  node[midway, right] {$T_{2y} =T_2 \sin\theta$} (B) ;



  % Angoli theta
  \draw (M) ++ (-0.75,0) arc[start angle=180, end angle=135, radius=0.75];
  \node at (-0.5,-2.8) {$\theta$};

  \draw (M) ++(0.75,0) arc[start angle=0, end angle=45, radius=0.75];
  \node at (0.5,-2.8) {$\theta$};
\end{tikzpicture}
    \caption{Oggetto sospeso a due cavi}
\end{figure}
\begin{equation*}
2 T \sin{45^\circ} = m \cdot g \Rightarrow T = \frac{m\cdot g}{2 \cdot \frac{\sqrt{2}}{2}} = \frac{m\cdot g}{\sqrt{2}}
\end{equation*}
\begin{equation*}
T = \frac{5\,\text{kg} \cdot \, 9.8\,\text{m/s}^2}{\sqrt{2}} = 34.6 \text{N}
\end{equation*}
\end{frame}

\begin{frame}
\frametitle{Esempio 3}
\textbf{Problema:} Un blocco di massa $8\,\text{kg}$ è su un piano inclinato di $20^\circ$. Determinare la forza normale esercitata dal piano sul blocco.
\textbf{Soluzione:}
\begin{equation*}
\vec{N} = m \cdot g \cdot \cos{\alpha} = 8\,\text{kg} \times 9.81\,\text{m/s}^2 \times \cos{20^\circ} = 73.68\,\text{N}
\end{equation*}
\end{frame}

\begin{frame}
\frametitle{Esempio 4}
\textbf{Problema:} Un oggetto di massa $15\,\text{kg}$ è sospeso da un cavo verticale. Calcolare la tensione nel cavo.
\begin{tikzpicture}[
    force/.style={>=latex,draw=blue,fill=blue},
    axis/.style={densely dashed, gray,font=\small},
    M/.style={rectangle,draw,fill=lightgray,minimum size=0.5cm,thin},
    m/.style={rectangle,draw=black,fill=lightgray,minimum size=0.3cm,thin},
    plane/.style={draw=black,fill=blue!10},
    string/.style={draw=red, thick},
    pulley/.style={thick},
]
    % Free body diagram of m
    \node[m] (m) {};
    \draw[axis,->] (m) -- ++(0,-2) node[left] {$+$};
    {[force,->]
        \draw (m.north) -- ++(0,1) node[above] {$T'$};
        \draw (m.south) -- ++(0,-1) node[right] {$\vec{P}$};
    }
\end{tikzpicture}

\textbf{Soluzione:}
\begin{equation*}
\vec{T} + \vec{P} = 0
\end{equation*}
\begin{equation*}
-T + P = 0 \Rightarrow P = T
\end{equation*}
\begin{equation*}
T = m \cdot g = 15\,\text{kg} \times 9.81\,\text{m/s}^2 = 147.15\,\text{N}
\end{equation*}
\end{frame}

\begin{frame}
\frametitle{Esempio 5}
\textbf{Problema:} Un corpo di massa $12\,\text{kg}$ è trainato su un piano orizzontale con una forza di $30\,\text{N}$ applicata con un angolo di $45^\circ$ rispetto all'orizzontale. Determinare la reazione vincolare verticale.
\begin{tikzpicture}
  % Disegna il piano orizzontale
  \draw[thick] (-1,0) -- (6,0);

  % Posizione del corpo
  \coordinate (A) at (2,0);

  % Disegna il corpo (un rettangolo)
  \draw[fill=blue!20] (A) rectangle ++(1,0.5);

  % Peso (forza gravitazionale)
  \draw[->, thick] ($(A)+(0.5,0)$) -- ++(0,-2) node[below] {$\vec{P} = mg$};

  % Forza applicata
  \draw[->, thick, red] ($(A)+(0.5,0.25)$) -- ++(1.5,1.5) node[ left] {$\vec{F}$};

  % Scomposizione della forza applicata
  % \draw[dashed, red] ($(A)+(0.5,0.25)$) -- ++(1.5,0);
  % \draw[dashed, red] ($(A)+(2.5,0.25)$) -- ++(0,1.5);

  % Componenti della forza applicata
  \draw[->, dashed, red] ($(A)+(0.5,0.25)$) -- ++(1.5,0) node[below right] {$F_x$};
  \draw[->, dashed, teal] ($(A)+(2,0.25)$) -- ++(0,1.5) node[right] {$F_y$};

  % Reazione vincolare
  \draw[->, thick, blue] ($(A)+(0.5,0.5)$) -- ++(0,0.5) node[above] {$\vec{N}$};

  % Angolo theta
  \draw ($(A)+(1.5,0.25)$) arc (0:45:1);
  \node at ($(A)+(1.2,0.5)$) {$\theta$};
\end{tikzpicture}

\textbf{Soluzione:}
\begin{equation*}
F_{y} = 30\,\text{N} \times \sin{45^\circ} = 21.21\,\text{N}
\end{equation*}
\begin{equation*}
\vec{N} = m \cdot g - F_{y} = 12\,\text{kg} \times 9.81\,\text{m/s}^2 - 21.21\,\text{N} = 96.51\,\text{N}
\end{equation*}
\end{frame}

\begin{frame}
\frametitle{Esempio 6}
\textbf{Problema:} Un piano inclinato ha un angolo di $60^\circ$. Qual è la forza normale su un blocco di $2\,\text{kg}$ posto su di esso?

\textbf{Soluzione:}

\begin{equation*}
\vec{N} = m \cdot g \cdot \cos{\alpha} = 2\,\text{kg} \times 9.81\,\text{m/s}^2 \times \cos{60^\circ} = 9.81\,\text{N}
\end{equation*}
\end{frame}

\begin{frame}
\frametitle{Esempio 7}
\textbf{Problema:} Un oggetto è in equilibrio sotto l'azione di tre forze. Due forze sono $\vec{F}_1 = 10\,\text{N}$ a $0^\circ$ e $\vec{F}_2 = 15\,\text{N}$ a $90^\circ$. Calcolare la terza forza $\vec{F}_3$ necessaria per mantenere l'equilibrio.

\textbf{Soluzione:}

\begin{equation*}
F_3 = \sqrt{10^2 + 15^2} = 18.03\,\text{N}
\end{equation*}
\begin{equation*}
\theta = \arctan{\left(\frac{15}{10}\right)} = 56.31^\circ
\end{equation*}
\end{frame}

\begin{frame}
\frametitle{Esempio 8}
\textbf{Problema:} Calcolare la tensione in due cavi che sostengono un carico di $20\,\text{kg}$, formando angoli di $30^\circ$ e $60^\circ$ con l'orizzontale.

\textbf{Soluzione:}

\begin{equation*}
T_2 = 170\,\text{N}
\end{equation*}
\begin{equation*}
T_1 = 98\,\text{N}
\end{equation*}
\end{frame}

\begin{frame}
\frametitle{Esempio 9}
\textbf{Problema:} Un oggetto è spinto su per un piano inclinato senza attrito con una forza parallela al piano. Se la forza è di $50\,\text{N}$, l'oggetto ha massa $5\,\text{kg}$ e l'angolo del piano è $30^\circ$, qual è l'accelerazione dell'oggetto?

\textbf{Soluzione:}

\begin{equation*}
F_{\text{netta}} = 50\,\text{N} - 24.525\,\text{N} = 25.475\,\text{N}
\end{equation*}
\begin{equation*}
a = \frac{25.475\,\text{N}}{5\,\text{kg}} = 5.095\,\text{m/s}^2
\end{equation*}
\end{frame}

\begin{frame}
\frametitle{Esempio 10}
\textbf{Problema:} Determinare la forza necessaria per mantenere in equilibrio un blocco su un piano inclinato di $25^\circ$, considerando un attrito con coefficiente $\mu = 0.1$. Il blocco ha una massa di $10\,\text{kg}$.

\textbf{Soluzione:}

\begin{equation*}
F_{\text{attrito}} = 0.1 \times 88.89\,\text{N} = 8.889\,\text{N}
\end{equation*}
\begin{equation*}
F = 41.42\,\text{N} - 8.889\,\text{N} = 32.53\,\text{N}
\end{equation*}
\end{frame}

\section{Esercizi}

\begin{frame}
\frametitle{Esercizi da Svolgere}
\begin{enumerate}
    \item Un blocco di massa $8\,\text{kg}$ è su un piano inclinato di $20^\circ$. Calcolare $\vec{P}_\parallel$ e $\vec{P}_\perp$.
    \item Un oggetto di $15\,\text{kg}$ è sospeso a un cavo. Determinare la tensione nel cavo.
    \item Un piano inclinato ha un angolo di $60^\circ$. Qual è la forza normale su un blocco di $2\,\text{kg}$?
    \item Un oggetto è in equilibrio sotto l'azione di tre forze. Due forze sono di $10\,\text{N}$ e $15\,\text{N}$ con angoli di $0^\circ$ e $90^\circ$. Calcolare la terza forza.
    \item Un corpo di massa $12\,\text{kg}$ è trainato su un piano orizzontale con una forza di $30\,\text{N}$ applicata con un angolo di $45^\circ$ rispetto all'orizzontale. Determinare la reazione vincolare.
    \item Calcolare la tensione in due cavi che sostengono un carico di $20\,\text{kg}$, formando angoli di $30^\circ$ e $60^\circ$ con l'orizzontale.
    \item Un blocco è in equilibrio su un piano inclinato senza attrito. Se l'angolo del piano è aumentato, cosa succede alle componenti della forza peso?
    \item Un oggetto è spinto su per un piano inclinato con una forza parallela al piano. Se la forza è di $50\,\text{N}$ e l'oggetto ha massa $5\,\text{kg}$, qual è l'accelerazione?
    \item Determinare la forza necessaria per mantenere in equilibrio un blocco su un piano inclinato di $25^\circ$ considerando attrito con coefficiente $\mu = 0.1$.
    \item Un ponte sostiene un carico distribuito uniformemente di $1000\,\text{kg}$. Calcolare la reazione vincolare totale del ponte.
\end{enumerate}
\end{frame}

\end{document}