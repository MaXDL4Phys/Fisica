\documentclass[a4paper,10pt]{article}
\usepackage[utf8]{inputenc}
\usepackage{amsmath}
\usepackage{geometry}

\geometry{a4paper, margin=1in}

\title{Esercizi di Problemi alla Fermi}
\author{Enrico Fermi (simulato)}
\date{\today}

\begin{document}

\maketitle

\section*{Introduzione}
I problemi alla Fermi richiedono di fare stime approssimative basate su informazioni limitate e ipotesi ragionevoli. L'obiettivo è ottenere una risposta con un ordine di grandezza accettabile, più che un risultato esatto. Di seguito sono elencati 10 problemi alla Fermi su cui esercitarsi.

\section*{Esercizi}
\begin{enumerate}
    \item \textbf{Quante palline da tennis servirebbero per riempire una piscina olimpionica?}
    
    \item \textbf{Quanto pesa l'acqua contenuta in un lago di medie dimensioni?}
    
    \item \textbf{Quanti cellulari sono venduti ogni anno nel mondo?}
    
    \item \textbf{Quante ore di musica sono ascoltate ogni giorno nel tuo paese?}
    
    \item \textbf{Quanto pesa tutta l'aria contenuta in un grattacielo?}
    
    \item \textbf{Quante pagine di carta sono usate ogni giorno in uffici di grandi città?}
    
    \item \textbf{Quanti litri di benzina consuma mediamente un aereo commerciale durante un volo transatlantico?}
    
    \item \textbf{Quanti pezzi di pizza sono mangiati in Italia ogni giorno?}
    
    \item \textbf{Quante formiche ci sono in un parco di 1 km²?}
    
    \item \textbf{Quanti litri d'acqua vengono usati per fare la doccia in un giorno in una grande città?}
\end{enumerate}

\section*{Istruzioni per gli Esercizi}
Per ciascun problema:
\begin{itemize}
    \item Identifica i fattori chiave necessari per calcolare la risposta.
    \item Fai ipotesi ragionevoli per ciascun fattore.
    \item Usa calcoli semplici per ottenere una stima finale.
    \item Confronta il tuo risultato con l'ordine di grandezza che ritieni plausibile.
\end{itemize}

\section*{Suggerimenti}
Ricorda che l'obiettivo non è ottenere una risposta precisa, ma piuttosto esercitarti nel pensiero critico e nell'approccio sistematico alla stima. Sii creativo e divertiti con i numeri!

\end{document}